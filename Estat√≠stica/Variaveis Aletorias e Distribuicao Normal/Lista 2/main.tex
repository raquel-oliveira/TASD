\documentclass[11pt,a4paper]{report}
\usepackage[brazil]{babel} %permite hifenação em português
%\usepackage[brazilian]{babel} %permite hifenação em português
\usepackage[utf8]{inputenc}
%\DeclareUnicodeCharacter{00A0}{ }
\usepackage{amssymb}
%simbolos da AMS
\usepackage{multicol}
\usepackage{latexsym}		%simbolos do Latex
\usepackage{amsfonts}		%fontes da AMS
\usepackage{bbm}		%simbolos de conjuntos
\usepackage{amsmath}
\usepackage{mathrsfs}
\usepackage{indentfirst}	%faz parágrafo na primeira linha
\usepackage{graphicx}		%inserção de gráficos ou figuras no texto
\usepackage{a4wide} 		%aumenta a área de uso do papel a4
\usepackage[all]{xy}		%pacote xypic
\usepackage[normalem]{ulem}
\usepackage{geometry}
\pagestyle{empty}
\usepackage[normalem]{ulem}
\setlength{\parindent}{2cm}	%tamanho de cada parágrafo
\usepackage{verbatim}
%\listfiles			%lista arquivos usados na compilação
%\setcounter{secnumdepth}{3}    %enumera até 3º nível de subseção

%%% para fazer o histograma
\usepackage{geometry}
\geometry{margin=1in} 
\usepackage{tikz}
\usepackage{tkz-euclide}
\usetikzlibrary{calc,intersections,through,backgrounds,snakes}
\usepackage{pgfplots}
\pgfplotsset{compat=1.8}
\usepgfplotslibrary{statistics}
\usepackage{bchart}
%%% fim dos packages necessrios para o histograma
%%% Text Companion Fonts
\usepackage{textcomp}

\newtheorem{teo}{Teorema}
\newtheorem{defi}[teo]{Definição}
\newtheorem{cor}[teo]{Corolário}
\newtheorem{lem}[teo]{Lema}
\newtheorem{pro}[teo]{Proposição}
\newtheorem{que}{Questão}
%\newtheorem{axi}[theorem]{Axioma}


\newenvironment{dem}[1][Demonstração]{\noindent\textbf{#1.} }{\ \rule{0.5em}{0.5em}}
\newenvironment{obs}[1][Observação]{\noindent\textbf{#1.} }{\ \rule{0.5em}{0.5em}}
\newenvironment{sol}[1][Solução]{\noindent\uline{#1:} }{}

\newcommand{\dprod}{\displaystyle\prod}
\newcommand{\dsum}{\displaystyle\sum}
\newcommand{\grupoq}[2]{\raisebox{0.1cm}{${#1}$}\!/\!\raisebox{-0.18cm}{${#2}$}}
\newcommand{\ex}[1]{\noindent{\bf  Exemplo {#1}}.\hspace*{0mm}}
\newcommand{\exs}[1]{{\bf  Exemplos {#1}}.\hspace*{0mm}}
\newcommand{\afir}[1]{{\bf Afirma\c c\~ao {#1}}:\hspace*{2mm}}
\newcommand{\afi}{{\bf Afirma\c c\~ao}:\hspace*{2mm}}
\newcommand{\F}{\mathbbmss{F}}
\newcommand{\N}{\mathbbmss{N}}
\newcommand{\Z}{\mathbbmss{Z}}
\newcommand{\R}{\mathbbmss{R}}
\newcommand{\Q}{\mathbbmss{Q}}
\newcommand{\C}{\mathbbmss{C}}
\newcommand{\Migual}{\geqslant}
\newcommand{\migual}{\leqslant}
\newcommand{\PP}{\mathscr{P}}
\newcommand{\CC}{\mathscr{C}}
\newcommand{\LL}{\mathscr{L}}
\newcommand{\fff}{{\cal F}}
\newcommand{\g}{{\cal G}}
\newcommand{\bbb}{{\cal B}}
\newcommand{\rrr}{I\!\!R}
\newcommand{\Pro}{\mathbb{P}}
\newcommand{\iid}{i.i.d.}
\newcommand{\pN}{\mathcal{N}}
\newcommand{\ddd}{\displaystyle}
\usepackage{color}
\usepackage{xcolor}
%\newcommand{\resposta}[1]{\textcolor{blue}{[RESPOSTA: #1]}}
\newcommand{\resposta}[1]{\textcolor{blue}{#1}}
%\usepackage[linguistics]{forest}
\usepackage{forest}

\geometry{textwidth=16cm,textheight=23.5cm}%escolhendo o espaço útil do texto.

\title{}
\author{}
\date{}

\begin{document}

\begin{center}

\textbf{Universidade Federal do Rio Grande do Norte\\
Instituto de Ciências Exatas e da Terra\\
Departamento de Estatística\\
Disciplina:} ESTATISTICA APLICADA A ENGENHARIA I\\
\textbf{Professora: }Kalline Fabiana Silveira\\
\textbf{Aluna: }Raquel Lopes de Oliveira\\
\vspace{0.5cm}
\textbf{Lista - Segunda Unidade}\vspace{0.5cm}

\end{center}


\begin{center}
\textbf{\LARGE{Lista Variáveis aleatórias}}
\end{center}
\begin{itemize}
	\item[\textbf{1.}] Determine a probabilidade de obtermos exatamente 3 caras em 6 lances de uma moeda.\\
		\resposta{A = probabilidade de ser cara\\
		$\Pro(A) = \frac{1}{2}$ e $\Pro(A)^c = \frac{1}{2}$\\
		Usa-se a distribuição binomial:\\
		\begin{align*}
		    \Pro(3) &= \frac{6!}{3!(6-3)!}(\frac{1}{2})^{3}(\frac{1}{2})^{6-3}\\
		    & = \frac{720}{36}\times0.125\times 0.125\\
		    &= 20 \times 0.015625\\
		    &= 0.3125
		\end{align*}}
	\item[\textbf{2.}] Jogando-se um dado três vezes, determine a probabilidade de se obter um múltiplo de 3 duas vezes\\
		\resposta{$\Omega = \{3,6\}$\\
	Y = ser um múltiplo de 3\\
	Z = não é multilo de 3\\
	Y'= Ser múltiplo de 3 duas vez\\
	$\Pro(Y) = \frac{2}{6} = \frac{1}{3}$\\$\Pro(Z) =\frac{2}{3}$
	\begin{align*}
	    \Pro(Y') &= \frac{3!}{2!(3-2)!}(\frac{1}{3})^2\times(\frac{2}{3})^1\\
	     &= 3 \times \frac{1}{9}\times\frac{2}{3}\\
	     &= \frac{2}{9}
	\end{align*}}
	\ifx Se a ordem dos dados importar, temos 3 permutacoes: (YYZ), (YZY), (ZYY), ou seja,\\
	\begin{align*}
	    \Pro(Y') &= \frac{2}{27}\times\frac{2}{27}\times\frac{2}{27}\\
	    &= \frac{2}{9}
	\end{align*}
	\fi
	\item[\textbf{3.}] Dois times de futebol, A e B, jogam entre si 6 vezes. Encontre a probabilidade do time A.
	\resposta{A = time A ganhar um jogo\\
	Supondo que num jogo A possa ou ganhar ou perder ou empatar, temos que:
	$\Pro(A) = \frac{1}{3}$}
	\begin{itemize}
	    \item[\textbf{a.}] ganhar dois ou três jogos\\
	    \resposta{
	    B = ganhar dois jogos\\
	    C = ganhar três jogos \\
	    D = ganhar dois ou tres jogos\\
	   \begin{align*}
	    \Pro(B) &= (\frac{1}{3})^2\times(\frac{2}{3})^4\\
	    &= \frac{1}{9}\times\frac{16}{81}\\
	    &= \frac{16}{729}\\
	    &= 0.0219
	  \end{align*}
	  Se a ordem importar então:    $\Pro(B) = C^6_2 \times 0.0219 = 15\times 0.0219 = 0.329$
	  \begin{align*}
	    \Pro(C) &= (\frac{1}{3})^3\times(\frac{2}{3})^3\\
	    &= \frac{1}{27}\times\frac{8}{27}\\
	    &= \frac{8}{729}\\
	    &= 0.0109
	  \end{align*}
	  Se a ordem importar então:    $\Pro(C) = C^6_3 \times 0.0109 = 20\times 0.0109 = 0.2194$\\
	  Sendo assim:
	  \begin{align*}
	      \Pro(D) &= \Pro(B) + \Pro(C)\\
	               &= 0.0219 + 0.0109 & \text{ordem nao importa}\\
	               &= 0.0328\\
	               ou\\
	               &= 0.329 + 0.2194 & \text{ordem importa}\\ 
	               &=0.5484        
	  \end{align*}}
	    \item[\textbf{b.}] ganhar pelo menos um jogo. \\
	    \resposta{A probabilidade de ganhar pelo menos um jogo corresponde a todas as possibilidades possíveis com exceção dele perder todos os jogos:\\
	    E = ganhar pelo menos um jogo\\
	    \begin{align*}
	        \Pro(E) &= 1 - C^6_0 \times(\frac{1}{3}) ^0\times(\frac{2}{3})^6\\
	        &= 1 - \frac{64}{729}\\
	        &= \frac{665}{729}\\
	        &= 0.912
	    \end{align*}}
	\end{itemize}
\newpage
	\item[\textbf{4.}] A probabilidade de um atirador acertar o alvo é 2/3. Se ele atirar 5 vezes, qual a probabilidade de acertar exatamente 2 tiros?\\
	\resposta{
	A = acertar dois tipos
	\begin{align*}
	    \Pro(A) &= C^5_2 \times (\frac{2}{3})^2\times(\frac{1}{3})^3\\
	    &= 10 \times \frac{4}{9} \times \frac{1}{27}\\
	    &= \frac{40}{243}\\
	    &= 0.164
	\end{align*}} 
	\item[\textbf{5.}] Seis parafusos são escolhidos ao acaso da produção de certa máquina, que apresenta 10\% de peças defeituosas. Qual a probabilidade de serem defeituosos dois deles?\\
	%	\duvida{a máquina tem 10\% de pecas defeituosas, mas eu nao possuo a probabilidade de um parafuso ser defeituoso...}\\
	%\resposta{Supondo que um parafuso tenha 10\% de ser defeituoso, então:\\
	\resposta{Tendo 10\% como a probabilidade de um parafuso ser defeituoso, então:\\
	\begin{align*}
	    \Pro(2) &= C^6_2\times(\frac{1}{10})^2\times(\frac{9}{10})^4\\
	    &= \frac{15\times6561}{1000000}\\
	    &=0.0984
	\end{align*}}
	\item[\textbf{6.}] No fichário de um hospital, estão arquivados os prontuários dos de 20 pacientes, que deram entrada no PS apresentando algum problema cardíaco. Destes 5 sofreram infarto. Retirando-se uma amostra ao acaso de 3 destes prontuários, qual a probabilidade de que dois deles sejam de pacientes que sofreram infarto? Calcule o Valor esperado e a variância.\\
	\resposta{N = número de itens da população= 20\\
	M = numero de itens da populacao que sao considerados como ``sucesso" =  5\\
	n = numero de itens na amostra = 3\\
	k = 2 (para as combinacoes)\\
	\begin{align*}
	    \Pro(X = k) &= \frac{C_k^MC_{n-k}^{N-M}}{C^N_n}\\
	    \Pro(2) &= \frac{C_2^{5}C_{1}^{15}}{C^{20}_{3}}\\
	    &= \frac{10\times15}{1140}\\
	    &=0.131578
	\end{align*}
	O valor esperado é: $\E(x) = \frac{m\times n}{N} = \frac{5 \times 3}{20} = 0.75$\\
	\ifx
    \begin{tabular}{c|c}
        X & $\Pro(X)$ \\
        \hline\hline
        0 & $\frac{C_0^{5}C_{3}^{15}}{C^{20}_{3}}$\\
        1 & $\frac{C_1^{5}C_{2}^{15}}{C^{20}_{3}}$\\
        2 & $\frac{C_2^{5}C_{1}^{15}}{C^{20}_{3}}$\\
        3 & $\frac{C_3^{5}C_{0}^{15}}{C^{20}_{3}}$ \\
        \hline\hline\\
        $\E(x)$ &
    \end{tabular}
    \fi
	A variância é:
	\begin{align*}
	    \V(X) &= M\times p\times q \times\frac{N-n}{N-1}\\
	    &= 5\times\frac{5}{20}\times\frac{15}{20}\times\frac{17}{19}\\
	    &= \frac{6375}{7600}\\
	    &= 08388
	\end{align*}} 
	\item[\textbf{7.}] Suponha que selecionemos aleatoriamente 5 cartas baralho sem reposição de um de um maço ordinário de jogo de baralho. Qual é a probabilidade de obter exatamente 2 cartas de baralho vermelhas (isto é, copas ou ouros)?\\
	\resposta{Para essa questão estou considerando o baralho lusófono que possui 52 cartas sendo 13 de cada naipe. Sendo assim, as cartas vermelhas são 26 dentre as 52.\\
	N = número de itens da população= 52\\
	M = numero de itens da populacao que sao considerados como ``sucesso" =  26\\
	n = numero de itens na amostra = 5\\
	k = 2 (para as combinacoes)\\
	\begin{align*}
	    \Pro(X = k) &= \frac{C_k^MC_{n-k}^{N-M}}{C^N_n}\\
	    \Pro(2) &= \frac{C_2^{26}C_{3}^{26}}{C^{52}_{5}}\\
	    &= \frac{325\times 2600}{2598960}\\
	    &=0.32513
	\end{align*}} 
	\item[\textbf{8.}] Numa Loteria, um apostador escolhe 6 números de 1 a 54. Qual a probabilidade dele acertar 5 números?\\
	\resposta{
	N = número de itens da população= 54\\
	M = numero de itens da populacao que sao considerados como ``sucesso" =  6\\
	n = numero de itens na amostra = 6\\
	k = 5 (para as combinacoes)\\
	\begin{align*}
	    \Pro(X = k) &= \frac{C_k^MC_{n-k}^{N-M}}{C^N_n}\\
	    \Pro(2) &= \frac{C_5^{6}C_{1}^{48}}{C^{54}_{6}}\\
	    &= \frac{6\times48}{25827165}\\
	    &= \frac{288}{25827165}\\
	    &= 0.000011151
	\end{align*}}  
	\item[\textbf{9.}] Suponha-se que haja 50 pessoas, dos quais 34 são MULHERES e o restante são HOMENS. Extrai-se uma amostra aleatória de 15 pessoas, sem reposição. Qual a probabilidade de exatamente 5 pessoas serem do sexo FEMININO?
	\resposta{} 
	
	\item[\textbf{10.}] 
	    \item[\textbf{a.}] Exatamente duas estejam queimadas?
	    \item[\textbf{b.}] Pelo menos uma esteja boa? \item[\textbf{c.}] Pelo menos duas estejam queimadas?
	    \item[\textbf{d.}] O número esperado de lâmpadas queimadas? \item[\textbf{e.}] A variância do número de lâmpadas queimadas?
	
\end{itemize}
\newpage
\begin{center}
\textbf{\LARGE{Distribuição Normal}}
\end{center}

\begin{itemize}
	\item[\textbf{1.}] Sejam X1 e X2 as v.a.’s que representam, respectivamente, os diâmetros do eixo e do soquete. Então $X_1 \pN( 3,42 ; 0,012 )$ e $X_2 \pN( 3,47 ; 0,022 ) $. Seja $Y =X_2 – X_1$. Suponha que, para efeitos de montagem, as componentes das peças são selecionadas ao acaso, e que eles só se encaixam se a folga estiver entre 0,025 cm e 0,100 cm. Qual a probabilidade do eixo se encaixar no soquete?\\
	\resposta{}
	
	
	\item[\textbf{2.}] A distribuição dos pesos de coelhos criados numa granja pode muito bem ser representada por uma distribuição Normal, com média 5 kg e desvio padrão 0,9 kg. Um abatedouro comprará 5000 coelhos e pretende classificá-los de acordo com o peso do seguinte modo: 15\% dos mais leves como pequenos, os 50\% seguintes como médios, os 20\% seguintes como grandes e os 15\% mais pesados como extras. Quais os limites de peso para cada classificação?
	\resposta{}
	 
	\item[\textbf{3.}] Sejam as variáveis normalmente distribuídas e independentes: $X1: \pN(100, 20)$ $X2: \pN(100, 30)$ e $X3: \pN(160, 40)$. Seja a variável Y calculada como sendo: $Y= 2X_1-X_2+3X_3$. Calcule:
		\begin{itemize}
		\item[a)]$\Pro(Y>590)$
	        \resposta{}
		\item[b)] $\Pro(Y<616)$
		    \resposta{}
		\item[c)] $\Pro(550<Y<570)$
		    \resposta{}
	\end{itemize}
	
	\item[\textbf{4.}] Considere 100 doadores escolhidos aleatoriamente de uma população onde a probabilidade de tipo A é 0,40? Qual a probabilidade de pelo menos 43 doadores terem sangue do tipo A?
	
	\item[\textbf{5.}] A taxa de desemprego em certa cidade é de 10\%. É obtida uma amostra aleatória de 100 pessoas. Qual a probabilidade de uma amostra ter, pelo menos, 15 pessoas desempregadas.
	
	\item[\textbf{6.}] Numa população, o peso dos indivíduos é uma variável aleatória X que segundo estudos anteriores segue o modelo normal com média 78 kg e desvio-padrão 10 kg. Uma pessoa é escolhida ao acaso nessa população. Determine a probabilidade de que seu peso:
		\begin{itemize}
		\item[a)] Seja maior que 60 kg;
	        \resposta{}
		\item[b)] Esteja entre 62kg e 72 kg
		    \resposta{}
		\item[c)] Seja inferior a 90 kg
		    \resposta{}
		\item[d)] Seja superior a 90 kg
		    \resposta{}
	\end{itemize}
	
\end{itemize}

\end{document}