\begin{center}
\textbf{\LARGE{Lista Variáveis aleatórias}}
\end{center}
\begin{itemize}
	\item[\textbf{1.}] Determine a probabilidade de obtermos exatamente 3 caras em 6 lances de uma moeda.
		\resposta{A = probabilidade de ser cara\\
		$\Pro(A) = \frac{1}{2}$ e $\Pro(A)^c = \frac{1}{2}$\\
		Usa-se a distribuição binomial:\\
		\begin{align*}
		    \Pro(3) &= \frac{6!}{3!(6-3)!}(\frac{1}{2})^{3}(\frac{1}{2})^{6-3}\\
		    & = \frac{720}{36}\times0.125\times 0.125\\
		    &= 20 \times 0.015625\\
		    &= 0.3125
		\end{align*}}
	\item[\textbf{2.}] Jogando-se um dado três vezes, determine a probabilidade de se obter um múltiplo de 3 duas vezes
	\resposta{} 
	\item[\textbf{3.}] Dois times de futebol, A e B, jogam entre si 6 vezes. Encontre a probabilidade do time A.
	\begin{itemize}
	    \item[\textbf{a.}] ganhar dois ou três jogos
	    \item[\textbf{b.}] ganhar pelo menos um jogo.
	\end{itemize}
	\item[\textbf{4.}] A probabilidade de um atirador acertar o alvo é 2/3. Se ele atirar 5 vezes, qual a probabilidade de acertar exatamente 2 tiros?
	\resposta{} 
	\item[\textbf{5.}] Seis parafusos são escolhidos ao acaso da produção de certa máquina, que apresenta 10\% de peças defeituosas. Qual a probabilidade de serem defeituosos dois deles?
	\resposta{} 
	\item[\textbf{6.}] No fichário de um hospital, estão arquivados os prontuários dos de 20 pacientes, que deram entrada no PS apresentando algum problema cardíaco. Destes 5 sofreram infarto. Retirando-se uma amostra ao acaso de 3 destes prontuários, qual a probabilidade de que dois deles sejam de pacientes que sofreram infarto? Calcule o Valor esperado e a variância.
	\resposta{} 
	\item[\textbf{7.}] Suponha que selecionemos aleatoriamente 5 cartas baralho sem reposição de um de um maço ordinário de jogo de baralho. Qual é a probabilidade de obter exatamente 2 cartas de baralho vermelhas (isto é, copas ou ouros)?
	\resposta{} 
	\item[\textbf{8.}] Numa Loteria, um apostador escolhe 6 números de 1 a 54. Qual a probabilidade dele acertar 5 números?
	\resposta{} 
	\item[\textbf{9.}] Suponha-se que haja 50 pessoas, dos quais 34 são MULHERES e o restante são HOMENS. Extrai-se uma amostra aleatória de 15 pessoas, sem reposição. Qual a probabilidade de exatamente 5 pessoas serem do sexo FEMININO?
	\resposta{} 
	
	\item[\textbf{10.}] 
	    \item[\textbf{a.}] Exatamente duas estejam queimadas?
	    \item[\textbf{b.}] Pelo menos uma esteja boa? \item[\textbf{c.}] Pelo menos duas estejam queimadas?
	    \item[\textbf{d.}] O número esperado de lâmpadas queimadas? \item[\textbf{e.}] A variância do número de lâmpadas queimadas?
	
\end{itemize}