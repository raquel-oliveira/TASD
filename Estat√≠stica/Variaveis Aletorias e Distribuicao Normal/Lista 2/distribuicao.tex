\begin{center}
\textbf{\LARGE{Distribuição Normal}}
\end{center}

\begin{itemize}
	\item[\textbf{1.}] Sejam X1 e X2 as v.a.’s que representam, respectivamente, os diâmetros do eixo e do soquete. Então $X_1 \pN( 3,42 ; 0,012 )$ e $X_2 \pN( 3,47 ; 0,022 ) $. Seja $Y =X_2 – X_1$. Suponha que, para efeitos de montagem, as componentes das peças são selecionadas ao acaso, e que eles só se encaixam se a folga estiver entre 0,025 cm e 0,100 cm. Qual a probabilidade do eixo se encaixar no soquete?\\
	
	\resposta{
	\begin{align*}
	    \E(Y) &= \E(X_2) - \E(X_1)\\
	    &= 3.47 - 3.42\\
	    &= 0.05
	\end{align*}
	\begin{align*}
	    \V(Y) &= \V(X_2) + \V(X_1)\\
	    &= 0.022 + 0.012\\
	    &= 0.034
	\end{align*}
	\begin{align*}
        Y &\sim \N(\E(Y), \V(Y))\\
        &\sim \N(0.05; 0.034)
    \end{align*}
	\begin{multicols}{2}
    \begin{align*}
      Z_1  &= \frac{0.025 - 0.05}{\sqrt{0.034}}\\
       &= -0.14
    \end{align*}
    \begin{align*}
       Z_2  &= \frac{0.1 - 0.05}{\sqrt{0.034}}\\
       &= 0.27
    \end{align*}
    \end{multicols}
    \begin{align*}
        \Pro(0.025 < y < 0.1) &= \Pro(0< Z < 0.14) + \Pro(0 < Z < 0.27)\\ &= 0.056 + 0.106\\
        &= 0.162
    \end{align*}}
	
	
	\item[\textbf{2.}] A distribuição dos pesos de coelhos criados numa granja pode muito bem ser representada por uma distribuição Normal, com média 5 kg e desvio padrão 0,9 kg. Um abatedouro comprará 5000 coelhos e pretende classificá-los de acordo com o peso do seguinte modo: 15\% dos mais leves como pequenos, os 50\% seguintes como médios, os 20\% seguintes como grandes e os 15\% mais pesados como extras. Quais os limites de peso para cada classificação? \\
	\resposta{X = peso de um coelho criado de uma granja\\
	$X\sim \N(5; 0,9^2)$\\
	$X_1$(pequenos) = 15\% mais leves que os demais\\
	$X_2$(pequenos e médios) = 65\% mais leves que os demais\\
	$X_3$(pequenos, médios e grandes) = 85\% mais leves que os demais\\
	\begin{align*}
	    \Pro(X < x_1) &= 0.15\\
	    \Pro(Z < \frac{x_1 - 5}{0.9} )&= 0.15\\
	    \frac{x_1 - 5}{0.9} &= -1.04\\
	    x_1 &= -1.04\times0.9 + 5\\
	    &= 4.064
	\end{align*}
	Ou seja, os coelhos que possuem um peso inferior a 4.064kg são considerados como pequenos
	\begin{align*}
	    \Pro(X < x_2) &= 0.65\\
	    \Pro(Z < \frac{x_1 - 5}{0.9} )&= 0.65\\
	    \frac{x_1 - 5}{0.9} &= 0.39\\
	    x_1 &= 0.39\times0.9 + 5\\
	    &= 5.351
	\end{align*}
	Ou seja, os coelhos que possuem um peso inferior a 5.351Kg e superior a 4.064kg são considerados como médios
	\begin{align*}
	    \Pro(X < x_1) &= 0.85\\
	    \Pro(Z < \frac{x_1 - 5}{0.9} )&= 0.85\\
	    \frac{x_1 - 5}{0.9} &= 1.04\\
	    x_1 &= 1.04\times0.9 + 5\\
	    &= 5.936
	\end{align*}
	Ou seja, os coelhos que possuem um peso inferior a 5.936Kg e superior a 5.351kg são considerados como grandes. E os superiores a 5.936Kg como extras.}

	\item[\textbf{3.}] Sejam as variáveis normalmente distribuídas e independentes: $X1: \pN(100, 20)$ $X2: \pN(100, 30)$ e $X3: \pN(160, 40)$. Seja a variável Y calculada como sendo: $Y= 2X_1-X_2+3X_3$. \\
	\resposta{\begin{align*}
	    \E(Y) &= 2\E(X_1) - \E(X_2) + 3\E(X_3)\\
	    &= 2\times100-100+3\times160\\
	    &= 100 +480\\
	    &= 580
	\end{align*}
	\begin{align*}
	    \V(Y) &= 2^2\V(X_1) + \V(X_2) + 3^2\V(X_3)\\
	    &= 80 + 30 + 360\\
	    &= 470
	\end{align*}
		\begin{align*}
        Y &\sim \N(\E(Y), \V(Y))\\
        &\sim \N(580; 470)
    \end{align*}}\\
	Calcule:
		\begin{itemize}
		\item[a)]$\Pro(Y>590)$\\
	        \resposta{
	        \begin{align*}
                Z  &= \frac{590 - 580}{\sqrt{470}}\\
                &= 0.461265604
            \end{align*}
             \begin{align*}
                \Pro(Y>590)  &= \Pro(Z > 0.461)\\
                &= 0,5 - \Pro(0\leq Z< 0.461)\\
                &= 0,5 - 0.1772\\
                &= 0.3228
            \end{align*}
        }
		\item[b)] $\Pro(Y<616)$\\
		    \resposta{
	        \begin{align*}
                Z  &= \frac{616 - 580}{\sqrt{470}}\\
                &= 1.660556174
            \end{align*}
            \begin{align*}
                \Pro(Y<616)  &= \Pro(Z < 1.66)\\
                &= 0.5 + \Pro(0 \leq Z < 1.66)\\
                &= 0.5 + 0.4515\\
                &= 0.9515
            \end{align*}
        }
		\item[c)] $\Pro(550<Y<570)$\\
		    \resposta{
		    \begin{multicols}{2}
	        \begin{align*}
                Z_1  &= \frac{550 - 580}{\sqrt{470}}\\
                &= -1.383796812
            \end{align*}
            \begin{align*}
                Z_2  &= \frac{570 - 580}{\sqrt{470}}\\
                &= -0.461265604
            \end{align*}
            \end{multicols}
            \begin{align*}
                \Pro(550<Y<570) &= \Pro(-1.383<Z<-0.4612)\\
               % &= \Pro(-1.383<Z\leq0) + \Pro(0\leq Z<-0.4612)\\
                &= \Pro(0\leq Z \leq 1.383) - \Pro(0\leq Z<-0.4612)\\
                &= 0.4162-0.1772\\
                &= 0.239
            \end{align*}
        }
	\end{itemize}
	
	\item[\textbf{4.}] Considere 100 doadores escolhidos aleatoriamente de uma população onde a probabilidade de tipo A é 0,40? Qual a probabilidade de pelo menos 43 doadores terem sangue do tipo A?
	
	\item[\textbf{5.}] A taxa de desemprego em certa cidade é de 10\%. É obtida uma amostra aleatória de 100 pessoas. Qual a probabilidade de uma amostra ter, pelo menos, 15 pessoas desempregadas.
	
	\item[\textbf{6.}] Numa população, o peso dos indivíduos é uma variável aleatória X que segundo estudos anteriores segue o modelo normal com média 78 kg e desvio-padrão 10 kg. Uma pessoa é escolhida ao acaso nessa população. Determine a probabilidade de que seu peso:
		\begin{itemize}
		\item[a)] Seja maior que 60 kg;
	        \resposta{}
		\item[b)] Esteja entre 62kg e 72 kg
		    \resposta{}
		\item[c)] Seja inferior a 90 kg
		    \resposta{}
		\item[d)] Seja superior a 90 kg
		    \resposta{}
	\end{itemize}
	
\end{itemize}