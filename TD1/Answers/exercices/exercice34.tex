\subsection*{Exercice 34 } %[CORRIGÉ 23 FÉVRIER]
\textit{Un central téléphonique possède L lignes. On estime à 1200 le nombre de personnes susceptibles d’appeler le standard sur une journée de 8 heures, la durée des ap- pels étant de deux minutes en moyenne.
On note $X$ la variable aléatoire égale au nombre de per- sonnes en train de téléphoner à un instant donné.
On suppose L = 3, calculer la probabilité d’encombre- ment à un instant donné, à savoir $\Pro(X > L)$.
Quelle doit être la valeur minimale de L pour qu’à un in- stant donné, la probabilité d’encombrement ne dépasse pas 0,1.}
\begin{align*}
    \text{L lignes} &= 3\\
    \text{1200 appels sur 8h}\\
    \text{duréé} &= 2\text{min}\\
    n &= 120 (2\text{min})\\
    X &= \text{personnes téléphonats}\\
    \Pro(X) &= 3600 = 1200 \times 3\\
    \text{Appels par seconde }&=
    \frac{1200}{8\times60\times60} = \frac{1}{8 \times 3} = \frac{1}{24} = P\\
    \text{On a bien n grad, p petit} \Rightarrow \lambda n \times p = \frac{120}{24} = 5\\
    \Pro(X >L) &= 1 - \Pro(X \leq L) \\
    &= 1 - (\Pro(X=0 \cup X =1 ...X=L))\\
    &\simeq 0,875\\
    \\
    \Pro(X >L) <0,1.\\
    \Pro(X>L) &= 1 - e^{- \lambda}(1+5+\frac{5^2}{25}+ \frac{5^3}{35}+ ...+ \frac{5^L}{L!})\\
    l = 3 \rightarrow  0,8 + 5\\
    L = \rightarrow 7 0,13\\
    L=8 \rightarrow  0,068
\end{align*}