\subsection*{Exercice 33}
\textit{Soit $X$ une variable aléatoire selon une loi de Poisson de paramètre $\lambda$ et $Y$ une variable aléatoire selon une loi de Poisson de paramètre \textmu{}. $X$ et $Y$ sont indépendantes. Soit $Z = X + Y$. Quelle est la loi de $Z$ ?}
%3-9 Distribuição Poisson
%http://www.cin.ufpe.br/~rmcrs/ESAP/arquivos/DistribuicoesBinomialPoisson.pdf

%A distribuição Poisson tem apenas um parâmetro, $\lambda$ que é interpretado como uma taxa média de ocorrência do evento
%http://leg.ufpr.br/~silvia/CE701/node35.html

"La somme de deux variables de Poisson indépendantes est également une variable de Poisson de paramètre égale à la somme de ses paramètres."\\
$K = \sum_{i=1}^n \lambda _i$        Cas générale
\\
\\
Donc: Z suit une loi de Poisson de paramètre $ \lambda + $ \textmu{}\\
Demonstration: \\
$X \sim \Pro(\lambda)$\\
$Y \sim \Pro(\mu{})$\\
$\Pro(X = i) = e^{-\lambda}\frac{-\lambda \lambda^i}{i!}$\\
$\Pro(Y = j) = e\frac{-\mu \mu^j}{j!}$\\

%Regarder: \url{http://www.jybaudot.fr/Probas/addipoisson.html}

