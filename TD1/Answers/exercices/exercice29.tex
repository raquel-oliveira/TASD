\subsection*{Exercice 29} %[CORRIGÉ 23 FÉVRIER]
\textit{Un joueur lance deux dés dont les faces sont numérotées de 1 à 6. On suppose que les dés sont non- truqués et donc que pour chaque dé, toutes les faces ont la même probabilité d’apparition. Le joueur suit les règles suivantes:}
\begin{itemize}
    \item \textit{Si les deux dés donnent le même numéro alors le joueur perd 10 points (A)}
    \item \textit{Si les deux dès donnent deux numéros de parités dif- férentes alors il perd 5 points (B)}
    \item \textit{Dans les autres cas il gagne 15 points.(C)}
\end{itemize}
\begin{align*}
    \text{Donc, le universe est:  } \Omega &= \{(d_1, d_2) \mid d_1 \in (1, 6] d_2 \in (1,6]\} \\
    \mid\Omega\mid &= 6 \times 6
\end{align*}
\begin{itemize}
    \item[(A)]  \textit{Le joueur joue une partie et on note X la variable alèatoire correspondant au nombre de points obtenus.
Déterminez la loi de probabilité de X puis calculez l’espérance de X.}
\begin{align*}
    \Pro(A) = \frac{6 \times1}{36} = \frac{1}{6}\\
    \Pro(B) = \frac{6 \times3}{36} = \frac{1}{2}\\
    \Pro(C) = 1 - (\frac{1}{6}+ \frac{1}{2}) = \frac{1}{3} \\
\end{align*}
\begin{align*}
    E(X) &= \sum_{w}X(w)\Pro(w)\\
    &= \sum_{i}i \times \Pro(X = i) \\
    &= 15 \times \Pro(X=15) -5 \times \Pro(X =-5) -10 \times \Pro(x=-10)\\
    &= \frac{2}{6}\times15 - \frac{3}{6}\times 5 - \frac{1}{6}\times 10\\
    &= \frac{30}{6} - \frac{15}{6} - \frac{10}{6}\\
    &= \frac{5}{6}
\end{align*}
    \item[(B)]  \textit{Le joueur effectue 10 parties de suites. Les résultats des parties sont indépendants les uns des autres. On appelle alors Y la variable aléatoire égale au nombre de fois que le joueur gagne 15 points.
Expliquez pourquoi Y suit une loi binomiale. Quels sont les paramètres de Y ?
Quelle est la probabilité que le joueur gagne au moins une fois 15 points ?
Combien de fois le joueur peut espérer gagner 15
points ?}
\begin{align*}
    \text{n: 10 parties}\\
    \text{y: a nombre de fois que C arrive}\\
    \text{y \til} B(n,p) \text{  p} = \frac{1}{3}
\end{align*}
\begin{align*}
    \Pro (y \geq 1) &= 1 - \Pro(y = 0)\\
    &= 1 - {\left( \begin{array}{cc}10 \\ 0\end{array} \right)}p^0(1-p)^{10-0}\\
    &=1-1\times1\times(\frac{2}{3})^10\\
    &\simeq 0,98
\end{align*}
\begin{align*}
    E(y) = n\times p = \frac{10}{3} \simeq 3,33\\
    0,999 = 1 - 10^{-4}
\end{align*}

\begin{align*} %definições
    B(n,p) : \Pro(X = k) = {\left( \begin{array}{cc}n \\ k\end{array} \right)}p^k(1-p)^{n-k}\\
    {\left( \begin{array}{cc}n \\ k\end{array} \right)} = \frac{n!}{k!(n-k)!} \tex{    and   }
    {\left( \begin{array}{cc}n \\ 0\end{array} \right)} = 1
\end{align*}
\begin{align*}
    \Pro(y \geq 1) &= 1 - {\left( \begin{array}{cc}n \\ 0\end{array} \right)}p^0(1-p)^n\\
    &= 1 - (\frac{2}{3})^n \geq 1 -10^4 \\
    &\Leftrightarrow (\frac{2}{3}) ^n \geq 10 ^{-4}  \text{ $\mid$   -2 $\geq$ 3 and 2 $\leq$ 3}\\
    &\Leftrightarrow e^{n ln(\frac{2}{3})} \leq e^{-k ln(10)}\\
    &\Leftrightarrow n\frac{ln\frac{2}{3}}{<0}\leq-kln(10)\\
    &\Leftrightarrow n \geq -4 \frac{ln 10}{ln \frac{2}{3}} = \frac{4 ln 10}{ln \frac{3}{2}}\\
    &n \geq 23 \text{parties}
\end{align*}
    \item[(C)]  \textit{Le joueur joue $n$ parties de suite. Quelle est la prob-
abilité qu’il gagne au moins une fois 15 points ?
A partir de quelle valeur de $n$ sa probabilité de gag- ner au moins une fois 15 points est strictement supérieure à 0,9999 ?}
\end{itemize}