\section{Considerações Iniciais}

\subsection{Definição de Corpo}

Um conjunto não vazio $\K$ é um \textbf{corpo}
se em seus elementos, estiverem definidas
duas operações, denominadas \textbf{adição} e \textbf{multiplicação}
que gozam das seguintes propriedades:

\begin{itemize}
\item[(C-A)] Para cada par $a$, $b$ de elementos de $\K$ corresponde um
$a + b \in V$, de modo que
\item[(C-A1)]
    $a + b = b + a, \forall{a, b} \in \K$
\item[(C-A2)]
    $a + (b + c) = (a + b) + c, \forall{a, b,c} \in \K $
\item[(C-A3)] Existe um elemento em $\K$, denotado por 0, tal que
    $ 0 + a = a, \forall{a} \in \K$
\item[(C-A4)] Para cada $a \in \K$, existe um elemento em K, denotado $-a$ tal que
    $a + (-a) = (-a) + a = 0$ 
    
\item [(C-M)] Para cada par $a$, $b$ de elementos de $\K$ corresponde um
$a \cdot b \in V$, de modo que
\item[(C-M1)]
    $a \cdot b = b \cdot a, \forall{a, b} \in \K$
\item[(C-M2)]
    $a \cdot (b \cdot c) = (a \cdot b) \cdot c, \forall{a, b, c} \in \K$
\item[(C-M3)] Existe um elemento em $\K$, denotado por 1, tal que 
    $ 1 \cdot a = a, \forall{a} \in \K$
\item[(C-M4)] Para cada elemento não nulo $a \in \K$, existe um elemento em $\K$, denotado por $a^{-1}$, tal que 
    $a^{-1} \cdot a = a \cdot a^{-1} = 1$ 
\item[(C-D)]
    $(a + b) \cdot c = a \cdot c + b \cdot c, \forall{a,b,c} \in \K$
\end{itemize}

%\newpage

\subsection{Definição de Espaço Vetorial}

Um conjunto não vazio $V$ é um \textbf{espaço vetorial sobre (um corpo) $\K$}
se em seus elementos, denominados \textbf{vetores}, estiverem definidas
duas operações, denominadas \textbf{soma} e \textbf{multiplicação por escalar}
(um elemento de $\K$) que gozam das seguintes propriedades:

\begin{itemize}
\item [(EV-A)] A cada par $\bm{u}$, $\bm{v}$ de vetores de $V$ corresponde um vetor
$\bm{u} + \bm{v} \in V$, chamado \textbf{soma de $\bm{u}$ e $\bm{v}$}, de modo que
\item [(EV-A1)] $\bm{u} + \bm{v} = \bm{v} + \bm{u}, \forall \bm{u},\bm{v} \in V$
\item [(EV-A2)] $(\bm{u} + \bm{v}) + \bm{w} = \bm{u} + (\bm{v} + \bm{w}), \forall \bm{u},\bm{v},\bm{w} \in V$
\item [(EV-A3)] Existe um vetor, denominado \textbf{vetor nulo}, e denotado por
$\bm{0}$, tal que $\bm{0} + \bm{v} = \bm{v}, \forall \bm{v} \in V$.
\item [(EV-A4)] Para cada vetor $\bm{v}$ em $ V$, existe um vetor em $V$,
denotado por $-\bm{v}$ e denominado simétrico ou negativo, tal que
$\bm{v} + (-\bm{v}) = (-\bm{v}) + \bm{v} = \bm{0}$.
\end{itemize}

\begin{itemize}
\item [(EV-M)] A cada par $\alpha \in \K$ e $\bm{v} \in V$ corresponde um
vetor $\alpha \cdot \bm{v} \in V$, chamado \textbf{produlo escalar de $\alpha$ por $\bm{v}$}, de modo que
\item [(EV-M1)] $(\alpha \cdot \beta)\cdot \bm{v} = \alpha \cdot (\beta \cdot \bm{v}), \forall \alpha, \beta \in \K \text{ e } \bm{v} \in V$
\item [(EV-M2)] $1 \cdot \bm{v} = \bm{v}, \forall \bm{v} \in V$
\end{itemize}

\begin{itemize}
\item [(EV-D1)] $\alpha \cdot (\bm{u} + \bm{v}) = \alpha \cdot \bm{u} + \alpha \cdot \bm{v}, \forall \alpha \in \K \text{ e } \forall \bm{u},\bm{v} \in V$.
\item [(EV-D2)] $(\alpha + \beta) \cdot \bm{v} = \alpha \cdot \bm{v} + \beta \cdot \bm{v}, \forall \alpha, \beta \in \K \text{ e } \forall \bm{v} \in V$.

\end{itemize}

\subsection{Proposições Auxiliares}

\begin{proposition} \label{leftnull} Sejam $V$ um espaço vetorial sobre $\K$,
$0$ o elemento neutro da adição em $\K$ e $\bm{0}$ o vetor nulo em $V$.
Para todo $\bm{v} \in V$, $0 \bm{v} = \bm{0}$.
\end{proposition}

\begin{proof}
\begin{align*}
  0 \bm{v} &= (0 + 0) \bm{v} \tag{C-A3} \\
  0 \bm{v} &= 0 \bm{v} + 0 \bm{v} \tag{EV-D2} \\
  0 \bm{v} + (-(0 \bm{v})) &= (0 \bm{v} + 0 \bm{v}) +(-(0 \bm{v})) \tag{EV-A} \\
  0 \bm{v} + (-(0 \bm{v})) &= 0 \bm{v} + (0 \bm{v} +(-(0 \bm{v}))) \tag{EV-A2} \\
  0 \bm{v} &= 0 \tag{EV-A4}
\end{align*}
\end{proof}

\begin{proposition} \label{rightnull} Sejam $V$ um espaço vetorial sobre $\K$,
e $\bm{0}$ o vetor nulo em $V$. Para todo $\alpha \in \K$, $\alpha \bm{0} = \bm{0}$.
\end{proposition}

\begin{proof}
\begin{align*}
  \alpha \bm{0} &= \alpha (\bm{0} + \bm{0}) \tag{EV-A3} \\
  \alpha \bm{0} &= \alpha \bm{0} + \alpha \bm{0} \tag{C-D2} \\
  \alpha \bm{0} + (-(\alpha \bm{0})) &= (\alpha \bm{0} + \alpha \bm{0}) + (-(\alpha \bm{0})) \tag{EV-A} \\
  \alpha \bm{0} + (-(\alpha \bm{0})) &= \alpha \bm{0} + (\alpha \bm{0} + (-(\alpha \bm{0}))) \tag{EV-A2} \\
  \alpha \bm{0} &= \bm{0} \tag{EV-A4}
\end{align*}
\end{proof}

\begin{proposition} \label{inverse} Seja $V$ um $\K$-espaço vetorial. $\forall \bm{v} \in V$, $-\bm{v} = (-1) \bm{v}$.
    \begin{proof}
        \begin{align*}
          \bm{v} + (-\bm{v}) &= \bm{0} \tag{EV-A4} \\
          1 \bm{v} + (-\bm{v}) &= \bm{0} \tag{EV-M2} \\
          (-1) \bm{v} + 1 \bm{v} + (-\bm{v}) &= (-1) \bm{v} + \bm{0}\\
          (-1) \bm{v} + 1 \bm{v} + (-\bm{v}) &= (-1) \bm{v} \tag{EV-A3}\\
          ((-1) + 1) \bm{v} + (-\bm{v}) &= (-1) \bm{v} \tag{EV-D2}\\
          0 \bm{v} + (-\bm{v}) &= (-1) \bm{v} \tag{C-A4}\\
          \bm{0} + (-\bm{v}) &= (-1) \bm{v} \tag{Proposição \ref{leftnull}}\\
          -\bm{v} &= (-1) \bm{v} \tag{EV-A3}
        \end{align*}
    \end{proof}
\end{proposition}