\documentclass[12pt,a4paper]{article}

\usepackage{amsmath,amsfonts,amssymb,amsthm}
\usepackage{color,graphicx,epsfig}
\usepackage{bm}

\usepackage[brazilian]{babel}
\usepackage[utf8]{inputenc}
\usepackage{verbatim}

\setlength{\parindent}{0in}
\setlength{\parskip}{2ex}

\usepackage[]{siunitx}
\usepackage{textcomp}
\usepackage{multicol}

\newcommand{\R}{\mathbb{R}}
%\newcommand{\N}{\mathbb{N}}
\newcommand{\N}{\mathcal{N}}
\newcommand{\Q}{\mathbb{Q}}
\newcommand{\K}{\mathbb{K}}
\newcommand{\Z}{\mathbb{Z}}
\newcommand{\Pro}{\mathbb{P}}
\newcommand\given[1][]{\:#1\vert\:}

\theoremstyle{plain}
\newtheorem{theorem}{Teorema}
\newtheorem{proposition}{Proposição}

\pagestyle{plain}

\begin{document}

\begin{center}
{\large\sc Polytech Nice Sophia} \\
{\large\sc SI3} \\

\vspace*{0.2cm}

{\sc Traitement et analyse statistique de données} \\
{Monsieur Theo Thonat }

\vspace*{0.2cm}
 Raquel Lopes de Oliveira

\vspace*{0.4cm}

{\large\bf  TADS - TD2 }

\end{center}

%\section{Considerações Iniciais}

\subsection{Definição de Corpo}

Um conjunto não vazio $\K$ é um \textbf{corpo}
se em seus elementos, estiverem definidas
duas operações, denominadas \textbf{adição} e \textbf{multiplicação}
que gozam das seguintes propriedades:

\begin{itemize}
\item[(C-A)] Para cada par $a$, $b$ de elementos de $\K$ corresponde um
$a + b \in V$, de modo que
\item[(C-A1)]
    $a + b = b + a, \forall{a, b} \in \K$
\item[(C-A2)]
    $a + (b + c) = (a + b) + c, \forall{a, b,c} \in \K $
\item[(C-A3)] Existe um elemento em $\K$, denotado por 0, tal que
    $ 0 + a = a, \forall{a} \in \K$
\item[(C-A4)] Para cada $a \in \K$, existe um elemento em K, denotado $-a$ tal que
    $a + (-a) = (-a) + a = 0$ 
    
\item [(C-M)] Para cada par $a$, $b$ de elementos de $\K$ corresponde um
$a \cdot b \in V$, de modo que
\item[(C-M1)]
    $a \cdot b = b \cdot a, \forall{a, b} \in \K$
\item[(C-M2)]
    $a \cdot (b \cdot c) = (a \cdot b) \cdot c, \forall{a, b, c} \in \K$
\item[(C-M3)] Existe um elemento em $\K$, denotado por 1, tal que 
    $ 1 \cdot a = a, \forall{a} \in \K$
\item[(C-M4)] Para cada elemento não nulo $a \in \K$, existe um elemento em $\K$, denotado por $a^{-1}$, tal que 
    $a^{-1} \cdot a = a \cdot a^{-1} = 1$ 
\item[(C-D)]
    $(a + b) \cdot c = a \cdot c + b \cdot c, \forall{a,b,c} \in \K$
\end{itemize}

%\newpage

\subsection{Definição de Espaço Vetorial}

Um conjunto não vazio $V$ é um \textbf{espaço vetorial sobre (um corpo) $\K$}
se em seus elementos, denominados \textbf{vetores}, estiverem definidas
duas operações, denominadas \textbf{soma} e \textbf{multiplicação por escalar}
(um elemento de $\K$) que gozam das seguintes propriedades:

\begin{itemize}
\item [(EV-A)] A cada par $\bm{u}$, $\bm{v}$ de vetores de $V$ corresponde um vetor
$\bm{u} + \bm{v} \in V$, chamado \textbf{soma de $\bm{u}$ e $\bm{v}$}, de modo que
\item [(EV-A1)] $\bm{u} + \bm{v} = \bm{v} + \bm{u}, \forall \bm{u},\bm{v} \in V$
\item [(EV-A2)] $(\bm{u} + \bm{v}) + \bm{w} = \bm{u} + (\bm{v} + \bm{w}), \forall \bm{u},\bm{v},\bm{w} \in V$
\item [(EV-A3)] Existe um vetor, denominado \textbf{vetor nulo}, e denotado por
$\bm{0}$, tal que $\bm{0} + \bm{v} = \bm{v}, \forall \bm{v} \in V$.
\item [(EV-A4)] Para cada vetor $\bm{v}$ em $ V$, existe um vetor em $V$,
denotado por $-\bm{v}$ e denominado simétrico ou negativo, tal que
$\bm{v} + (-\bm{v}) = (-\bm{v}) + \bm{v} = \bm{0}$.
\end{itemize}

\begin{itemize}
\item [(EV-M)] A cada par $\alpha \in \K$ e $\bm{v} \in V$ corresponde um
vetor $\alpha \cdot \bm{v} \in V$, chamado \textbf{produlo escalar de $\alpha$ por $\bm{v}$}, de modo que
\item [(EV-M1)] $(\alpha \cdot \beta)\cdot \bm{v} = \alpha \cdot (\beta \cdot \bm{v}), \forall \alpha, \beta \in \K \text{ e } \bm{v} \in V$
\item [(EV-M2)] $1 \cdot \bm{v} = \bm{v}, \forall \bm{v} \in V$
\end{itemize}

\begin{itemize}
\item [(EV-D1)] $\alpha \cdot (\bm{u} + \bm{v}) = \alpha \cdot \bm{u} + \alpha \cdot \bm{v}, \forall \alpha \in \K \text{ e } \forall \bm{u},\bm{v} \in V$.
\item [(EV-D2)] $(\alpha + \beta) \cdot \bm{v} = \alpha \cdot \bm{v} + \beta \cdot \bm{v}, \forall \alpha, \beta \in \K \text{ e } \forall \bm{v} \in V$.

\end{itemize}

\subsection{Proposições Auxiliares}

\begin{proposition} \label{leftnull} Sejam $V$ um espaço vetorial sobre $\K$,
$0$ o elemento neutro da adição em $\K$ e $\bm{0}$ o vetor nulo em $V$.
Para todo $\bm{v} \in V$, $0 \bm{v} = \bm{0}$.
\end{proposition}

\begin{proof}
\begin{align*}
  0 \bm{v} &= (0 + 0) \bm{v} \tag{C-A3} \\
  0 \bm{v} &= 0 \bm{v} + 0 \bm{v} \tag{EV-D2} \\
  0 \bm{v} + (-(0 \bm{v})) &= (0 \bm{v} + 0 \bm{v}) +(-(0 \bm{v})) \tag{EV-A} \\
  0 \bm{v} + (-(0 \bm{v})) &= 0 \bm{v} + (0 \bm{v} +(-(0 \bm{v}))) \tag{EV-A2} \\
  0 \bm{v} &= 0 \tag{EV-A4}
\end{align*}
\end{proof}

\begin{proposition} \label{rightnull} Sejam $V$ um espaço vetorial sobre $\K$,
e $\bm{0}$ o vetor nulo em $V$. Para todo $\alpha \in \K$, $\alpha \bm{0} = \bm{0}$.
\end{proposition}

\begin{proof}
\begin{align*}
  \alpha \bm{0} &= \alpha (\bm{0} + \bm{0}) \tag{EV-A3} \\
  \alpha \bm{0} &= \alpha \bm{0} + \alpha \bm{0} \tag{C-D2} \\
  \alpha \bm{0} + (-(\alpha \bm{0})) &= (\alpha \bm{0} + \alpha \bm{0}) + (-(\alpha \bm{0})) \tag{EV-A} \\
  \alpha \bm{0} + (-(\alpha \bm{0})) &= \alpha \bm{0} + (\alpha \bm{0} + (-(\alpha \bm{0}))) \tag{EV-A2} \\
  \alpha \bm{0} &= \bm{0} \tag{EV-A4}
\end{align*}
\end{proof}

\begin{proposition} \label{inverse} Seja $V$ um $\K$-espaço vetorial. $\forall \bm{v} \in V$, $-\bm{v} = (-1) \bm{v}$.
    \begin{proof}
        \begin{align*}
          \bm{v} + (-\bm{v}) &= \bm{0} \tag{EV-A4} \\
          1 \bm{v} + (-\bm{v}) &= \bm{0} \tag{EV-M2} \\
          (-1) \bm{v} + 1 \bm{v} + (-\bm{v}) &= (-1) \bm{v} + \bm{0}\\
          (-1) \bm{v} + 1 \bm{v} + (-\bm{v}) &= (-1) \bm{v} \tag{EV-A3}\\
          ((-1) + 1) \bm{v} + (-\bm{v}) &= (-1) \bm{v} \tag{EV-D2}\\
          0 \bm{v} + (-\bm{v}) &= (-1) \bm{v} \tag{C-A4}\\
          \bm{0} + (-\bm{v}) &= (-1) \bm{v} \tag{Proposição \ref{leftnull}}\\
          -\bm{v} &= (-1) \bm{v} \tag{EV-A3}
        \end{align*}
    \end{proof}
\end{proposition}


\section{Résolution des exercices}

\subsection*{Exercice 10 [CORRIGÉ 26 FÉVRIER]}
\textit{Le joueur A possède deux dés à six faces, et le joueur B possède un dé à douze faces (un dodécaèdre). Le joueur qui fait le plus grand score remporte la mise (match nul si égalité). Le jeu est-il équitable ? On calculera la proba- bilité que A gagne et la probabilité d’avoir un match nul.}\\
XA :2 dés à 6 faces\\
XB: dé à 12 faces\\
XA = $X_1 +X_2$
\begin{align*}
    \Pro(X_B = i) &= \frac{1}{12} | i \leq i \leq 12 \\
    \Pro(X_A = i) &= \Pro((X_1 + X_2 = i) \text{ ou } (X_1 =2, X_2 = i-2) ...)\\
    &= \sum_{k=1}^{i} \Pro(X_1 = k_i, X_2 = i-k)\\
    \Pro(X_1 = i) = \frac{1}{6} | 1 \leq i \leq 6
\end{align*}
$X_1$, $X_2$ : independants
\begin{align*}
\sum_{k=1}^i [\Pro(X_2 = k) \times \Pro(X_2 = i -k)]    
\end{align*}
\begin{align*}
    1 \leq k \leq 6 &\rightarrow k \leq max(i, 6)\\
    (1 \leq i-k \leq 6 &\rightarrow h++ \leq i \leq h+6 & h \geq i-6) 
\end{align*}
\begin{align*}
    &= \sum^{min(1,6)}_{max(1, i-6}[\Pro(X_1 = k) \times \Pro(X_2 = i -k) ]\\
    &= \frac{1}{36}\sum_{max(1,i-6)}^{min(i,6)}1
\end{align*}
\begin{align*}
    \text{Soit (f):} 2 \leq i \leq 7 &\Rightarrow min(i,6) =i \text{ et } max(i, i-6) =1\\
    \text{Soit (r):} 7 \leq i \leq 12 &\Rightarrow min(i,6) =6 \text{ et } max(i, i-6) = i = 6
\end{align*}
\begin{align*}
    (f) : \frac{1}{36}\sum_2^i1 = \frac{i-1}{36}\\
    (r) : \frac{1}{36}\sum_{i-6}^{12}1 = \frac{13 - i}{36}\\
    \Pro(X_A = i) = {\left\{ \begin{array}{cc}\frac{i-1}{36} \text{ si } 2 \leq i \leq 7\\ \frac{13-i}{36} \text{ si } 7\leq i \leq 12\end{array} \right\(}
\end{align*}
\begin{align*}
    \Pro(X_A > X_B) &= \Pro((X_B=1 \text{ et }x_A>1)\text{ ou } (X_B=2 \text{ et }x_A>2)....)\\
    &= \sum_{i=1}^{12}\Pro(X_B < i \text{ et }X_2=i)\\
     &= \sum_{i=1}^{12}\Pro(X_A = i) \sum_{k=1}^{i-1}\Pro(X_B = k)\\
     &= \frac{1}{12}\sum_{i=1}^{12}\Pro(X_A = i)\sum_{k=1}^{i-1}1\\
     &= \frac{1}{12}\sum_{i=1}^{12}(i-1)\Pro(X_A = i)\\
      &= \frac{1}{12}\sum_{i=1}^{12}i\Pro(X_A = i) - \frac{1}{12}\sum_{i=1}^{12}\Pro(X_A = i) = 1\\
       &= \frac{E(X_A)}{12} - \frac{1}{12} = \frac{6}{12} = \frac{1}{12}
\end{align*}


\begin{align*}
    \Pro(X_A > x_B) &= \frac{1}{2}\\
    \Pro(X_A = x_B) &= \frac{1}{12}\\
    \Pro(X_A < x_B) &= 1 - \frac{1}{2} - \frac{1}{12} = \frac{5}{12}
\end{align*}
\subsection*{Exercice 12}
\textit{Au marathon de New York 2000, 29 327 coureurs ont terminé. On suppose que le temps de parcours d’un coureur peut être approximé par une loi normale.}
\begin{itemize}
    \item[(A)]  \textit{ Sachant que le $10 000^e$ arrivant a mis 4h01’ et que le $1000^e$ a mis 3h08’, quelles sont la moyenne m et l’écart-type sigma des temps de parcours ?}\\
    \begin{align*}
        \text{Le } 10000^e &\rightarrow 4h01'&\rightarrow \Pro(X \leq 4h01) = \frac{10000}{29327}\\
        \text{Le } 1000^e &\rightarrow 3h08'&\rightarrow \Pro(X \leq 4h01) = \frac{10000}{29327}
    \end{align*}
    \begin{align*}
        \text{X: Temps de parcour d'un coureur}\\
        X \sim \N(\mu,\sigma)\\
        Y = \frac{X - \mu}{\sigma} \sim (0,1)
    \end{align*}
    \begin{align*}
        \Pro(X < 4h01) &= \Pro(Y \frac{4h01 - \mu}{\sigma}) &=\frac{10000}{29327} &= 0,34\\
        &= F(\frac{4h01 - \mu}{\sigma}) &&= 0,34
    \end{align*}
    \begin{align*}
        \frac{4h01 - \mu}{\sigma} &= F^{-1}(0,34) = -0,41\\
        \frac{3h08 - \mu}{\sigma} &= F^{-1}(0,034) = -1,8
    \end{align*}
    \begin{displaymath}
    \left\{ \begin{array}{ll}
    4h01 &= (-0,41 \times \sigma) + \mu\\
    3h07 &= (-1,8 \times \sigma) + \mu
    \end{array} \right.
    \left\{ \begin{array}{ll}
    \mu &= 4h01 + 0,41\sigma\\
    \mu &= 3h08 +1,8\sigma
    \end{array} \right.
    \end{displaymath}
    \begin{align*}
        \mu &= 241 + 0,4 \sigma\\
        \mu &= 188 + 1,8 \sigma\\
    \end{align*}
    \begin{align*}
        &241 + 0,4\sigma = 188 + 1,8\sigma\\
        \Leftrightarrow &241 -177 = -0,4\sigma + 1,8 \sigma\\
        \Leftrightarrow &53 = 139\sigma\\
        \Leftrightarrow &\sigma = \frac{53}{1,39} = 38min
    \end{align*}
        \begin{align*}
        \mu &= 241 + 0,41 \times 38\\
        \mu &= 256min\\
        \mu &= 4h16
    \end{align*}
    \item[(B)]\textit{Si vous aviez terminé en 3h48’, quel aurait été votre classement ?}
    \begin{align*}
        &\Pro(X \leq 3h48)\\
        =&\Pro(Y \leq \frac{3h48 - 4h16}{0h38})\\
        =&\Pro(Y \leq - 0,75)\\
        =&23,0\%
    \end{align*}
    Classement = $29.327 \times 0,23 = 6628^{eme}$
\end{document}