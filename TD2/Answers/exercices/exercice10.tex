\subsection*{Exercice 10 }
\textit{On suppose que la taille en centimètres d’un hu- main mâle de 25 ans suit une loi aléatoire normale de paramètre \textmu{}=175et $\sigma$=6.\\
Quel est le pourcentage des hommes de 25 ans ayant une taille supérieure à 185 cm ?}
\\
\begin{multicols}{2}
\begin{align*}
    X \sim &\Pro(\mu,\sigma)\\
    &\Pro(175, 6)
\end{align*}
\begin{align*}
    &Y = \frac{X - \mu}{\sigma} \sim \N(0,1)\\
    &X - \mu = y\sigma
    \\&X =  y\sigma + \mu
\end{align*}
\end{multicols}
\begin{align*}
    \Pro(X \geq 185) &= 1 - \Pro(x < 185)\\
    &= \Pro(X \leq 165)\\
    &=\Pro(y\sigma + \mu \leq 165)\\
    &=\Pro(y\leq \frac{165 - \mu}{\sigma})\\
    &= F(\frac{165 -  \mu}{\sigma})\\
    &= F( \frac{-10}{6})\\
    &= f(-1,67)\\
    &= 4,75 \%
\end{align*}
\textit{Parmi les hommes mesurant plus de 180 cm quel pour- centage mesure plus de 192 cm ?}\\
