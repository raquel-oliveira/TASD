\documentclass[12pt,a4paper]{article}

\usepackage{amsmath,amsfonts,amssymb,amsthm}
\usepackage{color,graphicx,epsfig}
\usepackage{bm}

\usepackage[brazilian]{babel}
\usepackage[utf8]{inputenc}
\usepackage{verbatim}

\setlength{\parindent}{0in}
\setlength{\parskip}{2ex}

\usepackage[]{siunitx}
\usepackage{textcomp}

\newcommand{\R}{\mathbb{R}}
\newcommand{\N}{\mathbb{N}}
\newcommand{\Q}{\mathbb{Q}}
\newcommand{\K}{\mathbb{K}}
\newcommand{\Z}{\mathbb{Z}}
\newcommand{\Pro}{\mathbb{P}}
\newcommand{\til}{\~{}}

\theoremstyle{plain}
\newtheorem{theorem}{Teorema}
\newtheorem{proposition}{Proposição}

\pagestyle{plain}

\begin{document}

\begin{center}
{\large\sc Polytech Nice Sophia} \\
{\large\sc SI3} \\

\vspace*{0.2cm}

{\sc Traitement et analyse statistique de données} \\
{Monsieur Theo Thonat }

\vspace*{0.2cm}
 Raquel Lopes de Oliveira

\vspace*{0.4cm}

{\large\bf  TADS - TD1 }

\end{center}

%\section{Considerações Iniciais}

\subsection{Definição de Corpo}

Um conjunto não vazio $\K$ é um \textbf{corpo}
se em seus elementos, estiverem definidas
duas operações, denominadas \textbf{adição} e \textbf{multiplicação}
que gozam das seguintes propriedades:

\begin{itemize}
\item[(C-A)] Para cada par $a$, $b$ de elementos de $\K$ corresponde um
$a + b \in V$, de modo que
\item[(C-A1)]
    $a + b = b + a, \forall{a, b} \in \K$
\item[(C-A2)]
    $a + (b + c) = (a + b) + c, \forall{a, b,c} \in \K $
\item[(C-A3)] Existe um elemento em $\K$, denotado por 0, tal que
    $ 0 + a = a, \forall{a} \in \K$
\item[(C-A4)] Para cada $a \in \K$, existe um elemento em K, denotado $-a$ tal que
    $a + (-a) = (-a) + a = 0$ 
    
\item [(C-M)] Para cada par $a$, $b$ de elementos de $\K$ corresponde um
$a \cdot b \in V$, de modo que
\item[(C-M1)]
    $a \cdot b = b \cdot a, \forall{a, b} \in \K$
\item[(C-M2)]
    $a \cdot (b \cdot c) = (a \cdot b) \cdot c, \forall{a, b, c} \in \K$
\item[(C-M3)] Existe um elemento em $\K$, denotado por 1, tal que 
    $ 1 \cdot a = a, \forall{a} \in \K$
\item[(C-M4)] Para cada elemento não nulo $a \in \K$, existe um elemento em $\K$, denotado por $a^{-1}$, tal que 
    $a^{-1} \cdot a = a \cdot a^{-1} = 1$ 
\item[(C-D)]
    $(a + b) \cdot c = a \cdot c + b \cdot c, \forall{a,b,c} \in \K$
\end{itemize}

%\newpage

\subsection{Definição de Espaço Vetorial}

Um conjunto não vazio $V$ é um \textbf{espaço vetorial sobre (um corpo) $\K$}
se em seus elementos, denominados \textbf{vetores}, estiverem definidas
duas operações, denominadas \textbf{soma} e \textbf{multiplicação por escalar}
(um elemento de $\K$) que gozam das seguintes propriedades:

\begin{itemize}
\item [(EV-A)] A cada par $\bm{u}$, $\bm{v}$ de vetores de $V$ corresponde um vetor
$\bm{u} + \bm{v} \in V$, chamado \textbf{soma de $\bm{u}$ e $\bm{v}$}, de modo que
\item [(EV-A1)] $\bm{u} + \bm{v} = \bm{v} + \bm{u}, \forall \bm{u},\bm{v} \in V$
\item [(EV-A2)] $(\bm{u} + \bm{v}) + \bm{w} = \bm{u} + (\bm{v} + \bm{w}), \forall \bm{u},\bm{v},\bm{w} \in V$
\item [(EV-A3)] Existe um vetor, denominado \textbf{vetor nulo}, e denotado por
$\bm{0}$, tal que $\bm{0} + \bm{v} = \bm{v}, \forall \bm{v} \in V$.
\item [(EV-A4)] Para cada vetor $\bm{v}$ em $ V$, existe um vetor em $V$,
denotado por $-\bm{v}$ e denominado simétrico ou negativo, tal que
$\bm{v} + (-\bm{v}) = (-\bm{v}) + \bm{v} = \bm{0}$.
\end{itemize}

\begin{itemize}
\item [(EV-M)] A cada par $\alpha \in \K$ e $\bm{v} \in V$ corresponde um
vetor $\alpha \cdot \bm{v} \in V$, chamado \textbf{produlo escalar de $\alpha$ por $\bm{v}$}, de modo que
\item [(EV-M1)] $(\alpha \cdot \beta)\cdot \bm{v} = \alpha \cdot (\beta \cdot \bm{v}), \forall \alpha, \beta \in \K \text{ e } \bm{v} \in V$
\item [(EV-M2)] $1 \cdot \bm{v} = \bm{v}, \forall \bm{v} \in V$
\end{itemize}

\begin{itemize}
\item [(EV-D1)] $\alpha \cdot (\bm{u} + \bm{v}) = \alpha \cdot \bm{u} + \alpha \cdot \bm{v}, \forall \alpha \in \K \text{ e } \forall \bm{u},\bm{v} \in V$.
\item [(EV-D2)] $(\alpha + \beta) \cdot \bm{v} = \alpha \cdot \bm{v} + \beta \cdot \bm{v}, \forall \alpha, \beta \in \K \text{ e } \forall \bm{v} \in V$.

\end{itemize}

\subsection{Proposições Auxiliares}

\begin{proposition} \label{leftnull} Sejam $V$ um espaço vetorial sobre $\K$,
$0$ o elemento neutro da adição em $\K$ e $\bm{0}$ o vetor nulo em $V$.
Para todo $\bm{v} \in V$, $0 \bm{v} = \bm{0}$.
\end{proposition}

\begin{proof}
\begin{align*}
  0 \bm{v} &= (0 + 0) \bm{v} \tag{C-A3} \\
  0 \bm{v} &= 0 \bm{v} + 0 \bm{v} \tag{EV-D2} \\
  0 \bm{v} + (-(0 \bm{v})) &= (0 \bm{v} + 0 \bm{v}) +(-(0 \bm{v})) \tag{EV-A} \\
  0 \bm{v} + (-(0 \bm{v})) &= 0 \bm{v} + (0 \bm{v} +(-(0 \bm{v}))) \tag{EV-A2} \\
  0 \bm{v} &= 0 \tag{EV-A4}
\end{align*}
\end{proof}

\begin{proposition} \label{rightnull} Sejam $V$ um espaço vetorial sobre $\K$,
e $\bm{0}$ o vetor nulo em $V$. Para todo $\alpha \in \K$, $\alpha \bm{0} = \bm{0}$.
\end{proposition}

\begin{proof}
\begin{align*}
  \alpha \bm{0} &= \alpha (\bm{0} + \bm{0}) \tag{EV-A3} \\
  \alpha \bm{0} &= \alpha \bm{0} + \alpha \bm{0} \tag{C-D2} \\
  \alpha \bm{0} + (-(\alpha \bm{0})) &= (\alpha \bm{0} + \alpha \bm{0}) + (-(\alpha \bm{0})) \tag{EV-A} \\
  \alpha \bm{0} + (-(\alpha \bm{0})) &= \alpha \bm{0} + (\alpha \bm{0} + (-(\alpha \bm{0}))) \tag{EV-A2} \\
  \alpha \bm{0} &= \bm{0} \tag{EV-A4}
\end{align*}
\end{proof}

\begin{proposition} \label{inverse} Seja $V$ um $\K$-espaço vetorial. $\forall \bm{v} \in V$, $-\bm{v} = (-1) \bm{v}$.
    \begin{proof}
        \begin{align*}
          \bm{v} + (-\bm{v}) &= \bm{0} \tag{EV-A4} \\
          1 \bm{v} + (-\bm{v}) &= \bm{0} \tag{EV-M2} \\
          (-1) \bm{v} + 1 \bm{v} + (-\bm{v}) &= (-1) \bm{v} + \bm{0}\\
          (-1) \bm{v} + 1 \bm{v} + (-\bm{v}) &= (-1) \bm{v} \tag{EV-A3}\\
          ((-1) + 1) \bm{v} + (-\bm{v}) &= (-1) \bm{v} \tag{EV-D2}\\
          0 \bm{v} + (-\bm{v}) &= (-1) \bm{v} \tag{C-A4}\\
          \bm{0} + (-\bm{v}) &= (-1) \bm{v} \tag{Proposição \ref{leftnull}}\\
          -\bm{v} &= (-1) \bm{v} \tag{EV-A3}
        \end{align*}
    \end{proof}
\end{proposition}
\section{Résolution des exercices}

%\input{exercices/exercice01.tex} 
%\subsection*{Exercice 2}
\textit{}
%\subsection*{Exercice 3}
\textit{}
%\subsection*{Exercice 4}
\textit{}
\subsection*{Exercice 5 [CORRIGÉ 2 FÉVRIER]}
\textit{On tire deux cartes d’un jeu de 52 cartes. Soit A l’événement ”les deux cartes ont la même valeur” et B l’événement ”les deux cartes ont la même couleur”. Etudier l’indépendance des événements A et B dans les cas suivants :}
\\
    \\
        $\forall{e} \in \Omega$   $\Pro(e) = \frac{1}{|\Omega|}$
    \\
    $A$ et $B$ sont independent, donc: $ \Pro(A \cap B) = \Pro(A) \times \Pro(B)$
\begin{itemize}
    \item[(A)]  \textit{on remet dans le jeu la première carte avant de tirer la seconde}
    \\
    $\Pro(A) = \frac{52 \times 4}{52^2} = \frac{1}{13}$
    \\
    \\
    $\Pro(B) = \frac{52 \times 13}{52^2} = \frac{1}{4}$
    \\
    \\
    $\Pro(A \cap B) = \frac{52 \times 1}{52^2} = \frac{1}{52}$
     \\
    \\
    $\Pro_B(A) = \frac{\Pro(B \cap A)}{\Pro(B)} = \frac{\frac{1}{52}}{\frac{1}{4}} = \frac{1}{13}$
    \\
    \\
    $\Pro(A \cup B) = \frac{1}{52} = \Pro(B)$
    
    \item[(B)]  \textit{on tire les deux cartes simultanément.}
    \\
    $\Omega = \{(a,b), a \neq b\} \\
    |\Omega| = 52 \times 51 $
    \\
    $\Pro(A) = \frac{52 \times 3}{52 \times 51} = \frac{3}{51}$
    \\
    \\
    $\Pro(B) = \frac{52 \times 12}{52 \times 51} = \frac{12}{51}$
    \\
    \\
    $\Pro(A \cap B) = \Pro(\{\}) = 0 \text{   Donc, don independants}$
    

\end{itemize} %02.02
%\subsection*{Exercice 6}
\textit{}
%\subsection*{Exercice 7}
\textit{}
\subsection*{Exercice 8 [CORRIGÉ 9 FÉVRIER]}
\textit{On considère trois cartes : une avec les deux faces rouges, une avec les deux faces blanches, et une avec une face rouge et une face blanche. On tire une carte au hasard. On expose une face au hasard. Elle est rouge. Parieriez-vous que la face cachée est blanche ? Pour vous aider dans votre choix :}
\begin{itemize}
    \item \textit{Déterminer l’espace de probabilité.}
    \begin{equation*}
        \Omega = \{\{\textbf{r}, b\}, \{r, \textbf{b}\}, \{\textbf{r},r\}, \{r,\textbf{r}\} \{\textbf{b}, b\}, \{b, \textbf{b}\}\}
    \end{equation*}
    \item \textit{Calculer la probabilité que la face cachée soit blanche sachant que la face visible est rouge.}\\
    A = \text{La face cachée est blanch} \\
    B = \text{La face visible est rouge}\\
    \begin{equation*}
        \Pro_B(A) = \frac{\Pro(A \cap B)}{\Pro(B)} = \frac{1/6}{1/2} = \frac{1}{3}
    \end{equation*}
\end{itemize}
%\subsection*{Exercice 9}
\textit{}
\subsection*{Exercice 10 [CORRIGÉ 26 FÉVRIER]}
\textit{Le joueur A possède deux dés à six faces, et le joueur B possède un dé à douze faces (un dodécaèdre). Le joueur qui fait le plus grand score remporte la mise (match nul si égalité). Le jeu est-il équitable ? On calculera la proba- bilité que A gagne et la probabilité d’avoir un match nul.}\\
XA :2 dés à 6 faces\\
XB: dé à 12 faces\\
XA = $X_1 +X_2$
\begin{align*}
    \Pro(X_B = i) &= \frac{1}{12} | i \leq i \leq 12 \\
    \Pro(X_A = i) &= \Pro((X_1 + X_2 = i) \text{ ou } (X_1 =2, X_2 = i-2) ...)\\
    &= \sum_{k=1}^{i} \Pro(X_1 = k_i, X_2 = i-k)\\
    \Pro(X_1 = i) = \frac{1}{6} | 1 \leq i \leq 6
\end{align*}
$X_1$, $X_2$ : independants
\begin{align*}
\sum_{k=1}^i [\Pro(X_2 = k) \times \Pro(X_2 = i -k)]    
\end{align*}
\begin{align*}
    1 \leq k \leq 6 &\rightarrow k \leq max(i, 6)\\
    (1 \leq i-k \leq 6 &\rightarrow h++ \leq i \leq h+6 & h \geq i-6) 
\end{align*}
\begin{align*}
    &= \sum^{min(1,6)}_{max(1, i-6}[\Pro(X_1 = k) \times \Pro(X_2 = i -k) ]\\
    &= \frac{1}{36}\sum_{max(1,i-6)}^{min(i,6)}1
\end{align*}
\begin{align*}
    \text{Soit (f):} 2 \leq i \leq 7 &\Rightarrow min(i,6) =i \text{ et } max(i, i-6) =1\\
    \text{Soit (r):} 7 \leq i \leq 12 &\Rightarrow min(i,6) =6 \text{ et } max(i, i-6) = i = 6
\end{align*}
\begin{align*}
    (f) : \frac{1}{36}\sum_2^i1 = \frac{i-1}{36}\\
    (r) : \frac{1}{36}\sum_{i-6}^{12}1 = \frac{13 - i}{36}\\
    \Pro(X_A = i) = {\left\{ \begin{array}{cc}\frac{i-1}{36} \text{ si } 2 \leq i \leq 7\\ \frac{13-i}{36} \text{ si } 7\leq i \leq 12\end{array} \right\(}
\end{align*}
\begin{align*}
    \Pro(X_A > X_B) &= \Pro((X_B=1 \text{ et }x_A>1)\text{ ou } (X_B=2 \text{ et }x_A>2)....)\\
    &= \sum_{i=1}^{12}\Pro(X_B < i \text{ et }X_2=i)\\
     &= \sum_{i=1}^{12}\Pro(X_A = i) \sum_{k=1}^{i-1}\Pro(X_B = k)\\
     &= \frac{1}{12}\sum_{i=1}^{12}\Pro(X_A = i)\sum_{k=1}^{i-1}1\\
     &= \frac{1}{12}\sum_{i=1}^{12}(i-1)\Pro(X_A = i)\\
      &= \frac{1}{12}\sum_{i=1}^{12}i\Pro(X_A = i) - \frac{1}{12}\sum_{i=1}^{12}\Pro(X_A = i) = 1\\
       &= \frac{E(X_A)}{12} - \frac{1}{12} = \frac{6}{12} = \frac{1}{12}
\end{align*}


\begin{align*}
    \Pro(X_A > x_B) &= \frac{1}{2}\\
    \Pro(X_A = x_B) &= \frac{1}{12}\\
    \Pro(X_A < x_B) &= 1 - \frac{1}{2} - \frac{1}{12} = \frac{5}{12}
\end{align*}
%\subsection*{Exercice 11 [CORRIGÉ 9 FÉVRIER]}
\textit{Les nouvelles plaques d’immatriculation des trotinettes sont formées de 6 lettres différentes de l’alphabet (A-Z). Chaque combinaison est équiprobable. Vous achetez une nouvelle trotinette. Quelle est la probabilité d’avoir les 6 lettres de votre plaque d’immatriculation dans l’ordre alphabétique ?}\\
\begin{equation*}
A < B < C ... < Z
\end{equation*}
ordre: 
$\left\{\begin{array}{l}
x_1, x_3 ... x_6 \\
\forall i , x_i < x_{i+1}
\end{array}$
\\


$25 \times 25 \times 24 \times 23 \times 22 \times 21 = \frac{26!}{20!}$ \\
\\
O = "La plaque est ordenée"\\
L = "La plaque cotinent les letres 'A''B', ...'F' " \\
$\Pro(L) = \frac{6!}{26! \times 20!} = \frac{6!20!}{26!} = \frac{1}{\left( \begin{array}{cc}
26 \\ 
6\end{array} \right)}$

\\
$\Pro_L(O) = \frac{\Pro(L \cap O)}{\Pro(L)} = \frac{1}{6!} = \Pro(O) $ car O et L sont independents$
%\subsection*{Exercice 12}
\textit{Au marathon de New York 2000, 29 327 coureurs ont terminé. On suppose que le temps de parcours d’un coureur peut être approximé par une loi normale.}
\begin{itemize}
    \item[(A)]  \textit{ Sachant que le $10 000^e$ arrivant a mis 4h01’ et que le $1000^e$ a mis 3h08’, quelles sont la moyenne m et l’écart-type sigma des temps de parcours ?}\\
    \begin{align*}
        \text{Le } 10000^e &\rightarrow 4h01'&\rightarrow \Pro(X \leq 4h01) = \frac{10000}{29327}\\
        \text{Le } 1000^e &\rightarrow 3h08'&\rightarrow \Pro(X \leq 4h01) = \frac{10000}{29327}
    \end{align*}
    \begin{align*}
        \text{X: Temps de parcour d'un coureur}\\
        X \sim \N(\mu,\sigma)\\
        Y = \frac{X - \mu}{\sigma} \sim (0,1)
    \end{align*}
    \begin{align*}
        \Pro(X < 4h01) &= \Pro(Y \frac{4h01 - \mu}{\sigma}) &=\frac{10000}{29327} &= 0,34\\
        &= F(\frac{4h01 - \mu}{\sigma}) &&= 0,34
    \end{align*}
    \begin{align*}
        \frac{4h01 - \mu}{\sigma} &= F^{-1}(0,34) = -0,41\\
        \frac{3h08 - \mu}{\sigma} &= F^{-1}(0,034) = -1,8
    \end{align*}
    \begin{displaymath}
    \left\{ \begin{array}{ll}
    4h01 &= (-0,41 \times \sigma) + \mu\\
    3h07 &= (-1,8 \times \sigma) + \mu
    \end{array} \right.
    \left\{ \begin{array}{ll}
    \mu &= 4h01 + 0,41\sigma\\
    \mu &= 3h08 +1,8\sigma
    \end{array} \right.
    \end{displaymath}
    \begin{align*}
        \mu &= 241 + 0,4 \sigma\\
        \mu &= 188 + 1,8 \sigma\\
    \end{align*}
    \begin{align*}
        &241 + 0,4\sigma = 188 + 1,8\sigma\\
        \Leftrightarrow &241 -177 = -0,4\sigma + 1,8 \sigma\\
        \Leftrightarrow &53 = 139\sigma\\
        \Leftrightarrow &\sigma = \frac{53}{1,39} = 38min
    \end{align*}
        \begin{align*}
        \mu &= 241 + 0,41 \times 38\\
        \mu &= 256min\\
        \mu &= 4h16
    \end{align*}
    \item[(B)]\textit{Si vous aviez terminé en 3h48’, quel aurait été votre classement ?}
    \begin{align*}
        &\Pro(X \leq 3h48)\\
        =&\Pro(Y \leq \frac{3h48 - 4h16}{0h38})\\
        =&\Pro(Y \leq - 0,75)\\
        =&23,0\%
    \end{align*}
    Classement = $29.327 \times 0,23 = 6628^{eme}$
%\subsection*{Exercice 13}
\textit{}
\subsection*{Exercice 14} %[CORRIGÉ 2 FÉVRIER]
\textit{On jette n fois une pièce de monnaie et on note $f_n$ le nombre de cas possibles où deux piles n’apparaissent pas successivement.}
\begin{itemize}
    \item[(A)]  \textit{Combien valent $f_1$ et $f_2$}
    \\
    \\
    $\Omega = \{\{p\},\{f\}\}$ \\ 
    $f_1 = 2$
    \\
    \\
    $\Omega = \{\{p,f\},\{f, p\}, \{f, f\},\{p, p\}\}$ \\ 
    $(f_2) = 3$ 
    \\
    \item[(B)]  \textit{Montrer que $f_n$ = $f_{n-1}$ + $f_{n-2}$}\\
    Suit recurrent lineaire:
    $I = f_3 = f_2 + f_2$ \\
    $\alpha^2 = \alpha + 1 \\
    \Delta = 5 \\
    \alpha_1 = \frac{1+ \sqrt{5}}{2} \\
    \alpha_2 = \frac{1- \sqrt{5}}{2} \\$
    
\begin{align*}
    %$\alpha_n &= A \alpha_1^n + B\alpha_2^n\\
    %\alpha_1 &= A \alpha_1 + B\alpha_2 = 2\\
    %\alpha_2 &= A \alpha_1^2 + B\alpha_2^2 = 3\\
    %&=  nn$
\end{align*}
    
    \item[(C)]  \textit{Calculer $f_n$ et la probabilité pour que sur n lancers il y ait au moins deux piles successifs.}
    \\
    \\
    $\Pro(p \geq 1) = 1 - \Pro(p < 1) = 1 -\sum_{i = 0}^{0}$ je sais pas
\end{itemize}
    
    
%\subsection*{Exercice 15}
\textit{}
%\subsection*{Exercice 16}
\textit{}
%\subsection*{Exercice 17 $[$CORRIGÉ 10 FÉVRIER$]$}
\textit{Au poker, on utilise un jeu de 52 cartes avec 4 couleurs de cartes (pique, trèfle, carreau, cœur). Une main comprend 5 cartes. Pour les différentes combinaisons, caLculer la probabilité de l’obtenir :
Une paire, une double paire, un brelan, un full, un carré, une quinte, une couleur, une quinte flush.}\\
\textbf{Definitions:}\\
1 paire: exatement 2 cartes de m valeurs\\
1 double paire: extement dux fois deux cartes de m valeurs avec 2 valeus different.\\
1 brelan: exatement 3 cartes de m valeurs
1 full: 3 cartes de une valeur e 2 caeres d'une autre valeur\\
1 carré:  exatement 4 cartes de m valeurs\\
1 quinte:5 cartes dant les valeurs se suivent (1-2-3-4-5-...-10-v-d-r-1)\\
1 couler: 5 cartes de meme coleurs. \\
1 quinte flush: 5 cartes de meme couler qui formment une suite.\\
1 quinte flush royale: 10-v-d-r-1 meme colour
\\
1 jeu de 52 cartes, 13 valeurs(1...10, V, D, R), 4 couleurs.\\
Un taie 5 cartes au hasard.\\
\begin{equation*}
    | \Omega |  = 
    {\left( \begin{array}{cc}52 \\ 5\end{array} \right)} =
    \frac{51!}{5!(52-5)!} = \frac{52 \times 51 \times 50 \times 49 \times 48}{5!} = 10 \text{valeurs de suite possibles}
\end{equation*}
Probabilité de obtenir un Color:\\
\begin{equation*}
    \Pro(Co) = \frac{{\left( \begin{array}{cc}4 \\ 1\end{array} \right)}{\left( \begin{array}{cc}13 \\ 5\end{array} \right)}}{{\left( \begin{array}{cc}52 \\ 5\end{array} \right)}} = \frac{33}{16660} = 0,19\%
\end{equation*}
Probabilité de obtenir un Pair:\\
\begin{align*}
    \Pro(P) &= \frac{{\left( \begin{array}{cc}13 \\ 1\end{array} \right)} \times {\left( \begin{array}{cc}4 \\ 2\end{array} \right)} \times {\left( \begin{array}{cc}12 \\ 1\end{array} \right)} \times {\left( \begin{array}{cc}4 \\ 1\end{array} \right)} \times {\left( \begin{array}{cc}11 \\ 1\end{array} \right)} \times {\left( \begin{array}{cc}4 \\ 1\end{array} \right)} \times {\left( \begin{array}{cc}10 \\ 1\end{array} \right)} \times {\left( \begin{array}{cc}4 \\ 1\end{array} \right)}}{{\left( \begin{array}{cc}52 \\ 5\end{array} \right)}}\\
    &= \frac{13 \times 4 \times 12 \times 4 \times 11 \times 4 \times 10 \times 4}{{\left( \begin{array}{cc}52 \\ 5\end{array} \right)}}\\
    &=42\%
\end{align*}
Probabilité de obtenir un Double Pair:\\
\begin{align*}
    \Pro(Dp) &= \frac{{\left( \begin{array}{cc}13 \\ 1\end{array} \right)} \times {\left( \begin{array}{cc}4 \\ 2\end{array} \right)} \times {\left( \begin{array}{cc}12 \\ 1\end{array} \right)} \times {\left( \begin{array}{cc}4 \\ 2\end{array} \right)} \times {\left( \begin{array}{cc}11 \\ 1\end{array} \right)} \times 4}{{\left( \begin{array}{cc}52 \\ 5\end{array} \right)}}\\
    &= \frac{13 \times \frac{4!}{2!2!} \times 12 \times \frac{4!}{2!2!} \times 11 \times 4}{{\left( \begin{array}{cc}52 \\ 5\end{array} \right)}}\\
    &= \frac{13 \tiimes 36 \times 12 \times 11 \times 4}{{\left( \begin{array}{cc}52 \\ 5\end{array} \right)}} \\
    &=4.75\%
\end{align*}
Probabilité de obtenir un Brelan:\\
$\Pro(B) = [\text{3 cartes de meme valeur}] \times [ \text{4eme carte} ] \times  [\text{5eme carte}$ ]\\
$\Pro(B) = [][][]$
\begin{align*}
    \Pro(B) &= {\left( \begin{array}{cc}13 \\ 1\end{array} \right)} \times {\left( \begin{array}{cc}4 \\ 3\end{array} \right)} \times {\left( \begin{array}{cc}12 \\ 1\end{array} \right)} \times {\left( \begin{array}{cc}4 \\ 1\end{array} \right)} \times {\left( \begin{array}{cc}11 \\ 1\end{array} \right)} \times {\left( \begin{array}{cc}4 \\ 1\end{array} \right)}\\
    &= \frac{13 \times 4 \times 12 \times 4 \times 11 \times 4}{{\left( \begin{array}{cc}52 \\ 5\end{array} \right)}}\\
    &=2.11\%
\end{align*}
Probabilité de obtenir un Full:\\
%$\Pro(F) = \text{1 value principal} \times \text{3 cartes} \times \text{} \times { text{}$\\
\begin{align*}
    \Pro(F) = {\left( \begin{array}{cc}13 \\ 1\end{array} \right)} \times {\left( \begin{array}{cc}4 \\ 3\end{array} \right)} \times {\left( \begin{array}{cc}12 \\ 1\end{array} \right)} \times {\left( \begin{array}{cc}4 \\ 2\end{array} \right)} &= \frac{13!}{1!12!} &\times \frac{4!}{1!} &\times \frac{12!}{11!} &\times \frac{4!}{2!2!} \\&= 13 &\times \frac{4!}{1!} &\times 12 &\times \frac{4!}{2!2!} \\&= \frac{13\times12\times(4!)^2}{4} \\
    &= 13 \time3\time(4!)^2 \\
    &= \frac{13\times3\times(4!)^2}{{\left( \begin{array}{cc}52 \\ 5\end{array} \right)}}
\end{align*}
Probabilité de obtenir un Carré:\\
\begin{equation*}
    \Pro(Ca) = \frac{{\left( \begin{array}{cc}13 \\ 1\end{array} \right)} \times {\left( \begin{array}{cc}12 \\ 1\end{array} \right)} \times 4}{{\left( \begin{array}{cc}52 \\ 3\end{array} \right)}} = \frac{13 \times 12 \times 4}{{\left( \begin{array}{cc}52 \\ 5\end{array} \right)}} = \frac{1}{4165} =* 0.024\%
    %aproximadamente
\end{equation*}
Probabilité de obtenir un Quinte:\\
\begin{equation*}
    \Pro(Q) = \frac{10\times4^5}{{\left( \begin{array}{cc}52 \\ 5\end{array} \right)}} = \frac{128}{32487} = 0.39\%
\end{equation*}
Probabilité de obtenir un Quinte Flush:\\
\begin{equation*}
    \Pro(Qf) = \frac{10 \times {\left( \begin{array}{cc}4 \\ 1\end{array} \right)}}{{\left( \begin{array}{cc}52 \\ 5\end{array} \right)}} = \frac{1}{64970} = 0.015\%
\end{equation*}
Probabilité de obtenir un Quinte Flush Royale:\\
\begin{equation*}
    \Pro(Qfr) = \frac{4 \times 1}{{\left( \begin{array}{cc}52 \\ 5\end{array} \right)}} = \frac{1}{649740} = 0.00015\%
\end{equation*}
Probabilité de obtenir AU MOINS un pair:\\
\begin{align*}
    \Pro(J) &= 1 - \frac{{\left( \begin{array}{cc}13 \\ 5\end{array} \right)} \times {{\left( \begin{array}{cc}4 \\ 1\end{array} \right)}}^5}{{\left( \begin{array}{cc}52 \\ 5\end{array} \right)}} \\
    &= 1 - \frac{\frac{13!}{5!8!} \times 4^5}{\frac{52!}{5!47!}} \\
    &= 1 - \frac{\frac{13!}{8!} \times 4^5 \times 47!}{52!}\\
    &= 1 - \frac{13 \times 12 \times 11 \times 10 \times 1}{52 \times 51 \times 50 \times 49 \times 48} \times 4^5
\end{align*}


%\subsection*{Exercice 18}
\textit{}
%\subsection*{Exercice 19}
\textit{}
%\subsection*{Exercice 20 }
\textit{}
\subsection*{Exercice 21 [CORRIGÉ 10 FÉVRIER]}
\textit{On suppose que l’on a autant de chance d’avoir une fille ou un garçon à la naissance. Votre voisin de palier vous dit qu’il a deux enfants.}\\
P = 0.5\\
$\Omega = \{\{F, G\}, \{G, G\}, \{F, F\}, \{G, F\}\}$
\begin{itemize}
    \item[(A)]  \textit{Quelle est la probabilité qu’il ait au moins un garçon ?}\\
    A: au moins un garçon.
    \begin{equation*}
        \Pro(A) = \frac{3}{4}
    \end{equation*}
    \item[(B)]  \textit{Quelle est la probabilité qu’il ait un garçon, sachant que l’aînée est une fille ?}\\
    B: l'aînée est un fille
    \begin{align*}
        \Pro(B) &= \frac{1}{2} \\
        \Pro(A \cap B) &= \frac{1}{4}\\
        \Pro_B(A) &= \frac{\Pro(A \cap B)}{\Pro(B)} \\
        &= \frac{\frac{1}{4}}{\frac{1}{2}} =\frac{1}{2}
    \end{align*}
    \item[(C)]  \textit{Quelle est la probabilité qu’il ait un garçon, sachant qu’il a au moins une fille ?}\\
    C: Au moins un fille
    \begin{align*}
        \Pro(C) &= \frac{3}{4} \\
        \Pro(A \cap C) &= \frac{1}{2}\\
        \Pro_C(A) &= \frac{\Pro(A \cap C)}{\Pro(C)} \\
        &= \frac{\frac{1}{2}}{\frac{3}{4}} = \frac{4}{6} = \frac{2}{3}
    \end{align*}
    \item[(D)]  \textit{Vous téléphonez à votre voisin. Une fille décroche le téléphone. Vous savez que dans les familles avec un garçon et une fille, la fille décroche le téléphone avec la probabilité p, quelle est la probabilité que votre voisin ait un garçon ?}\\
        D: une fille decroché
    
    \begin{align*}
        \Pro_{(F,G)}(D) &= \frac{\Pro(D \cap \{F,G\})}{\Pro(\{F,G\})} \\
        &= p\\
        \Pro(D \cap \{F,F\}) &= p \times \frac{1}{4}\\
        &= \frac{p}{4} \\
        \Pro(D) &= \Pro(D \cap \Omega)\\
        &= \Pro((D \cap \{F,F\}) \cup (D \cap \{G,G\}) \cup (D \cap \{F,G\}) \cup (D \cap\{G,F\}))
    \end{align*}
    \item[(E)]  \textit{Vous sonnez à la porte de votre voisin. Une fille ou- vre la porte. Sachant que l’aîné(e) ouvre la porte avec probabilité p, et ce indépendamment de la réparti- tion de la famille, quelle est la probabilité que votre voisin ait un garçon ?}\\
    E: Une fille ouvre la porte
    \begin{align*}
        \Pro_E(A) = \frac{\Pro(A \cap E)}{\Pro(E)}
    \end{align*}
    \end{itemize}
%\subsection*{Exercice 22}
\textit{ Le jeu du 35
On lance 5 billes dans 3 trous. Chaque bille est indépen- dante des autres et a autant de chance d’arriver dans l’un des trois trous. Le gain du joueur est le suivant : soit $n_1$ le nombre de billes dans le trou qui a le plus de bille, $n_3$ le nombre de billes dans le trou qui a le moins de bille. Alors le gain est :}
\begin{equation*}
   2 \times n_1 - n_3 - 5
\end{equation*}

\begin{itemize}
    \item[(A)]  \textit{Montrer qu’il n’existe que 5 configurations finales dans ce jeu.}
    \item[(B)]  \textit{Calculer pour chaque fin possible le gain du joueur et la probabilité de cet évènement.}
    \item[(C)]  \textit{Calculer l’espérance du gain et expliquer s’il est in- téressant de jouer à ce jeu ou non.}
\end{itemize}
%\subsection*{Exercice 23 }
\textit{}
%\subsection*{Exercice 24}
\textit{}
%\subsection*{Exercice 25}
\textit{}
%\subsection*{Exercice 26}
\textit{}
\subsection*{Exercice 27 [CORRIGÉ 10 FÉVRIER]}
\textit{Supposons que la probabilité qu’un enfant soit de sexe féminin est 0.4. Quelle est la probabilité d’avoir, parmi 5 enfants, au moins un garcon et au moins une fille ?}\\
X: le nb de filles \\
X \til B(5; 0.4)\\
$\Pro(X = i) = {\left( \begin{array}{cc}5 \\ i\end{array} \right)} \times p^i\times(1-)^{5 - i}$\\
A = Avoir au moins une fille $\Leftrightarrow$ x $\geq$ 1 \\
B = Avoir au moin un garçon $\Leftrightarrow$ 5 - x $\geq$ 1 $\Leftrightarrow$ x $\leq$ 4
\begin{align*}
    \Pro(A \cap B) &\\
    \Pro(1 \leq X \leq 4) &= 1 - \Pro(X=0 \text{ou} X =5)\\
    &= 1 - ({\left( \begin{array}{cc}5 \\ 0\end{array} \right)}0.4^0(1-0.4)^{ 5-0}) + ({\left( \begin{array}{cc}5 \\ 5\end{array} \right)}0.4^5(1-0.4)^{5-5}) )\\
    &=((1\times1\times0.6^5)+(1\times0.4^5\times1))\\
    &=1- (0.6^5 + 0.4^5)\\
    &= 1 - 0.088 \\
    & = 0.912
\end{align*}
%\subsection*{Exercice 28}
\textit{Un écran d’ordinateur est formé de pixels. Il comporte 768 lignes de 1024 pixels.
On utilise un procédé de fabrication qui assure que les pixels sont indépendants et que chacun n’a qu’une probabilité $9.10^{-7}$ d’être inutilisable. Quelle est la loi du nombre X de pixels grillés sur l’écran et son espérance ?} 
\subsection*{Exercice 29} %[CORRIGÉ 23 FÉVRIER]
\textit{Un joueur lance deux dés dont les faces sont numérotées de 1 à 6. On suppose que les dés sont non- truqués et donc que pour chaque dé, toutes les faces ont la même probabilité d’apparition. Le joueur suit les règles suivantes:}
\begin{itemize}
    \item \textit{Si les deux dés donnent le même numéro alors le joueur perd 10 points (A)}
    \item \textit{Si les deux dès donnent deux numéros de parités dif- férentes alors il perd 5 points (B)}
    \item \textit{Dans les autres cas il gagne 15 points.(C)}
\end{itemize}
\begin{align*}
    \text{Donc, le universe est:  } \Omega &= \{(d_1, d_2) \mid d_1 \in (1, 6] d_2 \in (1,6]\} \\
    \mid\Omega\mid &= 6 \times 6
\end{align*}
\begin{itemize}
    \item[(A)]  \textit{Le joueur joue une partie et on note X la variable alèatoire correspondant au nombre de points obtenus.
Déterminez la loi de probabilité de X puis calculez l’espérance de X.}
\begin{align*}
    \Pro(A) = \frac{6 \times1}{36} = \frac{1}{6}\\
    \Pro(B) = \frac{6 \times3}{36} = \frac{1}{2}\\
    \Pro(C) = 1 - (\frac{1}{6}+ \frac{1}{2}) = \frac{1}{3} \\
\end{align*}
\begin{align*}
    E(X) &= \sum_{w}X(w)\Pro(w)\\
    &= \sum_{i}i \times \Pro(X = i) \\
    &= 15 \times \Pro(X=15) -5 \times \Pro(X =-5) -10 \times \Pro(x=-10)\\
    &= \frac{2}{6}\times15 - \frac{3}{6}\times 5 - \frac{1}{6}\times 10\\
    &= \frac{30}{6} - \frac{15}{6} - \frac{10}{6}\\
    &= \frac{5}{6}
\end{align*}
    \item[(B)]  \textit{Le joueur effectue 10 parties de suites. Les résultats des parties sont indépendants les uns des autres. On appelle alors Y la variable aléatoire égale au nombre de fois que le joueur gagne 15 points.
Expliquez pourquoi Y suit une loi binomiale. Quels sont les paramètres de Y ?
Quelle est la probabilité que le joueur gagne au moins une fois 15 points ?
Combien de fois le joueur peut espérer gagner 15
points ?}
\begin{align*}
    \text{n: 10 parties}\\
    \text{y: a nombre de fois que C arrive}\\
    \text{y \til} B(n,p) \text{  p} = \frac{1}{3}
\end{align*}
\begin{align*}
    \Pro (y \geq 1) &= 1 - \Pro(y = 0)\\
    &= 1 - {\left( \begin{array}{cc}10 \\ 0\end{array} \right)}p^0(1-p)^{10-0}\\
    &=1-1\times1\times(\frac{2}{3})^10\\
    &\simeq 0,98
\end{align*}
\begin{align*}
    E(y) = n\times p = \frac{10}{3} \simeq 3,33\\
    0,999 = 1 - 10^{-4}
\end{align*}

\begin{align*} %definições
    B(n,p) : \Pro(X = k) = {\left( \begin{array}{cc}n \\ k\end{array} \right)}p^k(1-p)^{n-k}\\
    {\left( \begin{array}{cc}n \\ k\end{array} \right)} = \frac{n!}{k!(n-k)!} \tex{    and   }
    {\left( \begin{array}{cc}n \\ 0\end{array} \right)} = 1
\end{align*}
\begin{align*}
    \Pro(y \geq 1) &= 1 - {\left( \begin{array}{cc}n \\ 0\end{array} \right)}p^0(1-p)^n\\
    &= 1 - (\frac{2}{3})^n \geq 1 -10^4 \\
    &\Leftrightarrow (\frac{2}{3}) ^n \geq 10 ^{-4}  \text{ $\mid$   -2 $\geq$ 3 and 2 $\leq$ 3}\\
    &\Leftrightarrow e^{n ln(\frac{2}{3})} \leq e^{-k ln(10)}\\
    &\Leftrightarrow n\frac{ln\frac{2}{3}}{<0}\leq-kln(10)\\
    &\Leftrightarrow n \geq -4 \frac{ln 10}{ln \frac{2}{3}} = \frac{4 ln 10}{ln \frac{3}{2}}\\
    &n \geq 23 \text{parties}
\end{align*}
    \item[(C)]  \textit{Le joueur joue $n$ parties de suite. Quelle est la prob-
abilité qu’il gagne au moins une fois 15 points ?
A partir de quelle valeur de $n$ sa probabilité de gag- ner au moins une fois 15 points est strictement supérieure à 0,9999 ?}
\end{itemize}
%\subsection*{Exercice 30 }
\textit{}
%\input{exercices/exercice31.tex}
%\input{exercices/exercice32.tex}
%\subsection*{Exercice 33}
\textit{Soit $X$ une variable aléatoire selon une loi de Poisson de paramètre $\lambda$ et $Y$ une variable aléatoire selon une loi de Poisson de paramètre \textmu{}. $X$ et $Y$ sont indépendantes. Soit $Z = X + Y$. Quelle est la loi de $Z$ ?}
%3-9 Distribuição Poisson
%http://www.cin.ufpe.br/~rmcrs/ESAP/arquivos/DistribuicoesBinomialPoisson.pdf

%A distribuição Poisson tem apenas um parâmetro, $\lambda$ que é interpretado como uma taxa média de ocorrência do evento
%http://leg.ufpr.br/~silvia/CE701/node35.html

"La somme de deux variables de Poisson indépendantes est également une variable de Poisson de paramètre égale à la somme de ses paramètres."\\
$K = \sum_{i=1}^n \lambda _i$        Cas générale
\\
\\
Donc: Z suit une loi de Poisson de paramètre $ \lambda + $ \textmu{}\\
Demonstration: \\
$X \sim \Pro(\lambda)$\\
$Y \sim \Pro(\mu{})$\\
$\Pro(X = i) = e^{-\lambda}\frac{-\lambda \lambda^i}{i!}$\\
$\Pro(Y = j) = e\frac{-\mu \mu^j}{j!}$\\

%Regarder: \url{http://www.jybaudot.fr/Probas/addipoisson.html}

 
\subsection*{Exercice 34 } %[CORRIGÉ 23 FÉVRIER]
\textit{Un central téléphonique possède L lignes. On estime à 1200 le nombre de personnes susceptibles d’appeler le standard sur une journée de 8 heures, la durée des ap- pels étant de deux minutes en moyenne.
On note $X$ la variable aléatoire égale au nombre de per- sonnes en train de téléphoner à un instant donné.
On suppose L = 3, calculer la probabilité d’encombre- ment à un instant donné, à savoir $\Pro(X > L)$.
Quelle doit être la valeur minimale de L pour qu’à un in- stant donné, la probabilité d’encombrement ne dépasse pas 0,1.}
\begin{align*}
    \text{L lignes} &= 3\\
    \text{1200 appels sur 8h}\\
    \text{duréé} &= 2\text{min}\\
    n &= 120 (2\text{min})\\
    X &= \text{personnes téléphonats}\\
    \Pro(X) &= 3600 = 1200 \times 3\\
    \text{Appels par seconde }&=
    \frac{1200}{8\times60\times60} = \frac{1}{8 \times 3} = \frac{1}{24} = P\\
    \text{On a bien n grad, p petit} \Rightarrow \lambda n \times p = \frac{120}{24} = 5\\
    \Pro(X >L) &= 1 - \Pro(X \leq L) \\
    &= 1 - (\Pro(X=0 \cup X =1 ...X=L))\\
    &\simeq 0,875\\
    \\
    \Pro(X >L) <0,1.\\
    \Pro(X>L) &= 1 - e^{- \lambda}(1+5+\frac{5^2}{25}+ \frac{5^3}{35}+ ...+ \frac{5^L}{L!})\\
    l = 3 \rightarrow  0,8 + 5\\
    L = \rightarrow 7 0,13\\
    L=8 \rightarrow  0,068
\end{align*}
%\input{exercices/exercice35.tex}
%\input{exercices/exercice36.tex}
%\input{exercices/exercice37.tex}
%\input{exercices/exercice38.tex} 
%\subsection*{Exercice 39}
\textit{Soit X une variable aléatoire discrète qui suit une loi
géométrique. Montrer que
$\forall n, k \in \N, \Pro(\{X=n+k\}/\{X>k\})=\Pro(X=n)$}\\
\\
$X \sim \text{loi trigonométrique de parametres} \\
\forall n, k, \Pro(X = n + k | x > k) = P (x = n) \\
\Pro (X = i)  = (1-p) ^{i-1}p\\
\Pro(A|B) = P_B(A)          \text{//Section A.4}\\
\Pro_B(A) = \frac{P(A \cap B)}{\Pro(B)} \\$
\begin{align*}
   $ \Pro (x = n+k| x > k) &= \frac{\Pro(X = n+k \cap x > k)}{\Pro(x>k)}\\
   &= \frac{\Pro(X = n+k)}{\Pro(X>k)}\\
   \Pro(x=n+k| x >k) &=\frac{(A-p)^{n+k-1}p}{\Pro(X=k+1 \cup X=k+2 \cup ...) & \text{//Section A.2}\\
   \sum_{i=k+1}^{+\infty}\Pro(X=i) \\
   &= \frac{(1-p)^{n+k-1} p}{\sum_{i=k+1}^{+\infty}(1-p)^{i-1} p$
\end{align*}

%\input{exercices/exercice40.tex} 

\end{document}
