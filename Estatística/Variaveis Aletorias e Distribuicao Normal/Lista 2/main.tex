\documentclass[11pt,a4paper]{report}
\usepackage[brazil]{babel} %permite hifenação em português
%\usepackage[brazilian]{babel} %permite hifenação em português
\usepackage[utf8]{inputenc}
%\DeclareUnicodeCharacter{00A0}{ }
\usepackage{amssymb}
%simbolos da AMS
\usepackage{multicol}
\usepackage{latexsym}		%simbolos do Latex
\usepackage{amsfonts}		%fontes da AMS
\usepackage{bbm}		%simbolos de conjuntos
\usepackage{amsmath}
\usepackage{mathrsfs}
\usepackage{indentfirst}	%faz parágrafo na primeira linha
\usepackage{graphicx}		%inserção de gráficos ou figuras no texto
\usepackage{a4wide} 		%aumenta a área de uso do papel a4
\usepackage[all]{xy}		%pacote xypic
\usepackage[normalem]{ulem}
\usepackage{geometry}
\pagestyle{empty}
\usepackage[normalem]{ulem}
\setlength{\parindent}{2cm}	%tamanho de cada parágrafo
\usepackage{verbatim}
%\listfiles			%lista arquivos usados na compilação
%\setcounter{secnumdepth}{3}    %enumera até 3º nível de subseção

%%% para fazer o histograma
\usepackage{geometry}
\geometry{margin=1in} 
\usepackage{tikz}
\usepackage{tkz-euclide}
\usetikzlibrary{calc,intersections,through,backgrounds,snakes}
\usepackage{pgfplots}
\pgfplotsset{compat=1.8}
\usepgfplotslibrary{statistics}
\usepackage{bchart}
%%% fim dos packages necessrios para o histograma
%%% Text Companion Fonts
\usepackage{textcomp}

\newtheorem{teo}{Teorema}
\newtheorem{defi}[teo]{Definição}
\newtheorem{cor}[teo]{Corolário}
\newtheorem{lem}[teo]{Lema}
\newtheorem{pro}[teo]{Proposição}
\newtheorem{que}{Questão}
%\newtheorem{axi}[theorem]{Axioma}


\newenvironment{dem}[1][Demonstração]{\noindent\textbf{#1.} }{\ \rule{0.5em}{0.5em}}
\newenvironment{obs}[1][Observação]{\noindent\textbf{#1.} }{\ \rule{0.5em}{0.5em}}
\newenvironment{sol}[1][Solução]{\noindent\uline{#1:} }{}

\newcommand{\dprod}{\displaystyle\prod}
\newcommand{\dsum}{\displaystyle\sum}
\newcommand{\grupoq}[2]{\raisebox{0.1cm}{${#1}$}\!/\!\raisebox{-0.18cm}{${#2}$}}
\newcommand{\ex}[1]{\noindent{\bf  Exemplo {#1}}.\hspace*{0mm}}
\newcommand{\exs}[1]{{\bf  Exemplos {#1}}.\hspace*{0mm}}
\newcommand{\afir}[1]{{\bf Afirma\c c\~ao {#1}}:\hspace*{2mm}}
\newcommand{\afi}{{\bf Afirma\c c\~ao}:\hspace*{2mm}}
\newcommand{\E}{\mathbbmss{E}}
\newcommand{\F}{\mathbbmss{F}}
\newcommand{\N}{\mathbbmss{N}}
\newcommand{\Z}{\mathbbmss{Z}}
\newcommand{\R}{\mathbbmss{R}}
\newcommand{\Q}{\mathbbmss{Q}}
\newcommand{\C}{\mathbbmss{C}}
\newcommand{\V}{\mathbbmss{V}}
\newcommand{\Migual}{\geqslant}
\newcommand{\migual}{\leqslant}
\newcommand{\PP}{\mathscr{P}}
\newcommand{\CC}{\mathscr{C}}
\newcommand{\LL}{\mathscr{L}}
\newcommand{\fff}{{\cal F}}
\newcommand{\g}{{\cal G}}
\newcommand{\bbb}{{\cal B}}
\newcommand{\rrr}{I\!\!R}
\newcommand{\Pro}{\mathbb{P}}
\newcommand{\iid}{i.i.d.}
\newcommand{\pN}{\mathcal{N}}
\newcommand{\ddd}{\displaystyle}
\usepackage{color}
\usepackage{multicol}
\usepackage{xcolor}
%\newcommand{\resposta}[1]{\textcolor{blue}{[RESPOSTA: #1]}}
\newcommand{\resposta}[1]{\textcolor{blue}{#1}}
\newcommand{\duvida}[1]{\textcolor{red}{DUVIDA: #1}}
%\usepackage[linguistics]{forest}
\usepackage{forest}

\geometry{textwidth=16cm,textheight=23.5cm}%escolhendo o espaço útil do texto.

\title{}
\author{}
\date{}

\begin{document}

\begin{center}

\textbf{Universidade Federal do Rio Grande do Norte\\
Instituto de Ciências Exatas e da Terra\\
Departamento de Estatística\\
Disciplina:} ESTATISTICA APLICADA A ENGENHARIA I\\
\textbf{Professora: }Kalline Fabiana Silveira\\
\textbf{Aluna: }Raquel Lopes de Oliveira\\
\vspace{0.5cm}
\textbf{Lista - Segunda Unidade}\vspace{0.5cm}

\end{center}


\begin{center}
\textbf{\LARGE{Lista Variáveis aleatórias}}
\end{center}
\begin{itemize}
	\item[\textbf{1.}] Determine a probabilidade de obtermos exatamente 3 caras em 6 lances de uma moeda.
	\resposta{} 
	\item[\textbf{2.}] Jogando-se um dado três vezes, determine a probabilidade de se obter um múltiplo de 3 duas vezes
	\resposta{} 
	\item[\textbf{3.}] Dois times de futebol, A e B, jogam entre si 6 vezes. Encontre a probabilidade do time A.
	\begin{itemize}
	    \item[\textbf{a.}] ganhar dois ou três jogos
	    \item[\textbf{b.}] ganhar pelo menos um jogo.
	\end{itemize}
	\item[\textbf{4.}] A probabilidade de um atirador acertar o alvo é 2/3. Se ele atirar 5 vezes, qual a probabilidade de acertar exatamente 2 tiros?
	\resposta{} 
	\item[\textbf{5.}] Seis parafusos são escolhidos ao acaso da produção de certa máquina, que apresenta 10\% de peças defeituosas. Qual a probabilidade de serem defeituosos dois deles?
	\resposta{} 
	\item[\textbf{6.}] No fichário de um hospital, estão arquivados os prontuários dos de 20 pacientes, que deram entrada no PS apresentando algum problema cardíaco. Destes 5 sofreram infarto. Retirando-se uma amostra ao acaso de 3 destes prontuários, qual a probabilidade de que dois deles sejam de pacientes que sofreram infarto? Calcule o Valor esperado e a variância.
	\resposta{} 
	\item[\textbf{7.}] Suponha que selecionemos aleatoriamente 5 cartas baralho sem reposição de um de um maço ordinário de jogo de baralho. Qual é a probabilidade de obter exatamente 2 cartas de baralho vermelhas (isto é, copas ou ouros)?
	\resposta{} 
	\item[\textbf{8.}] Numa Loteria, um apostador escolhe 6 números de 1 a 54. Qual a probabilidade dele acertar 5 números?
	\resposta{} 
	\item[\textbf{9.}] Suponha-se que haja 50 pessoas, dos quais 34 são MULHERES e o restante são HOMENS. Extrai-se uma amostra aleatória de 15 pessoas, sem reposição. Qual a probabilidade de exatamente 5 pessoas serem do sexo FEMININO?
	\resposta{} 
	
	\item[\textbf{10.}] 
	    \item[\textbf{a.}] Exatamente duas estejam queimadas?
	    \item[\textbf{b.}] Pelo menos uma esteja boa? \item[\textbf{c.}] Pelo menos duas estejam queimadas?
	    \item[\textbf{d.}] O número esperado de lâmpadas queimadas? \item[\textbf{e.}] A variância do número de lâmpadas queimadas?
	
\end{itemize}
\newpage
\begin{center}
\textbf{\LARGE{Distribuição Normal}}
\end{center}

\begin{itemize}
	\item[\textbf{1.}] Sejam X1 e X2 as v.a.’s que representam, respectivamente, os diâmetros do eixo e do soquete. Então $X_1 \pN( 3,42 ; 0,012 )$ e $X_2 \pN( 3,47 ; 0,022 ) $. Seja $Y =X_2 – X_1$. Suponha que, para efeitos de montagem, as componentes das peças são selecionadas ao acaso, e que eles só se encaixam se a folga estiver entre 0,025 cm e 0,100 cm. Qual a probabilidade do eixo se encaixar no soquete?\\
	
	\resposta{
	\begin{align*}
	    \E(Y) &= \E(X_2) - \E(X_1)\\
	    &= 3.47 - 3.42\\
	    &= 0.05
	\end{align*}
	\begin{align*}
	    \V(Y) &= \V(X_2) + \V(X_1)\\
	    &= 0.022 + 0.012\\
	    &= 0.034
	\end{align*}
	\begin{align*}
        Y &\sim \N(\E(Y), \V(Y))\\
        &\sim \N(0.05; 0.034)
    \end{align*}
	\begin{multicols}{2}
    \begin{align*}
      Z_1  &= \frac{0.025 - 0.05}{\sqrt{0.034}}\\
       &= -0.14
    \end{align*}
    \begin{align*}
       Z_2  &= \frac{0.1 - 0.05}{\sqrt{0.034}}\\
       &= 0.27
    \end{align*}
    \end{multicols}
    \begin{align*}
        \Pro(0.025 < y < 0.1) &= \Pro(0< Z < 0.14) + \Pro(0 < Z < 0.27)\\ &= 0.056 + 0.106\\
        &= 0.162
    \end{align*}}
	
	
	\item[\textbf{2.}] A distribuição dos pesos de coelhos criados numa granja pode muito bem ser representada por uma distribuição Normal, com média 5 kg e desvio padrão 0,9 kg. Um abatedouro comprará 5000 coelhos e pretende classificá-los de acordo com o peso do seguinte modo: 15\% dos mais leves como pequenos, os 50\% seguintes como médios, os 20\% seguintes como grandes e os 15\% mais pesados como extras. Quais os limites de peso para cada classificação? \\
	\resposta{X = peso de um coelho criado de uma granja\\
	$X\sim \N(5; 0,9^2)$\\
	$X_1$(pequenos) = 15\% mais leves que os demais\\
	$X_2$(pequenos e médios) = 65\% mais leves que os demais\\
	$X_3$(pequenos, médios e grandes) = 85\% mais leves que os demais\\
	\begin{align*}
	    \Pro(X < x_1) &= 0.15\\
	    \Pro(Z < \frac{x_1 - 5}{0.9} )&= 0.15\\
	    \frac{x_1 - 5}{0.9} &= -1.04\\
	    x_1 &= -1.04\times0.9 + 5\\
	    &= 4.064
	\end{align*}
	Ou seja, os coelhos que possuem um peso inferior a 4.064kg são considerados como pequenos
	\begin{align*}
	    \Pro(X < x_2) &= 0.65\\
	    \Pro(Z < \frac{x_1 - 5}{0.9} )&= 0.65\\
	    \frac{x_1 - 5}{0.9} &= 0.39\\
	    x_1 &= 0.39\times0.9 + 5\\
	    &= 5.351
	\end{align*}
	Ou seja, os coelhos que possuem um peso inferior a 5.351Kg e superior a 4.064kg são considerados como médios
	\begin{align*}
	    \Pro(X < x_1) &= 0.85\\
	    \Pro(Z < \frac{x_1 - 5}{0.9} )&= 0.85\\
	    \frac{x_1 - 5}{0.9} &= 1.04\\
	    x_1 &= 1.04\times0.9 + 5\\
	    &= 5.936
	\end{align*}
	Ou seja, os coelhos que possuem um peso inferior a 5.936Kg e superior a 5.351kg são considerados como grandes. E os superiores a 5.936Kg como extras.}

	\item[\textbf{3.}] Sejam as variáveis normalmente distribuídas e independentes: $X1: \pN(100, 20)$ $X2: \pN(100, 30)$ e $X3: \pN(160, 40)$. Seja a variável Y calculada como sendo: $Y= 2X_1-X_2+3X_3$. \\
	\resposta{\begin{align*}
	    \E(Y) &= 2\E(X_1) - \E(X_2) + 3\E(X_3)\\
	    &= 2\times100-100+3\times160\\
	    &= 100 +480\\
	    &= 580
	\end{align*}
	\begin{align*}
	    \V(Y) &= 2^2\V(X_1) + \V(X_2) + 3^2\V(X_3)\\
	    &= 80 + 30 + 360\\
	    &= 470
	\end{align*}
		\begin{align*}
        Y &\sim \N(\E(Y), \V(Y))\\
        &\sim \N(580; 470)
    \end{align*}}\\
	Calcule:
		\begin{itemize}
		\item[a)]$\Pro(Y>590)$\\
	        \resposta{
	        \begin{align*}
                Z  &= \frac{590 - 580}{\sqrt{470}}\\
                &= 0.461265604
            \end{align*}
             \begin{align*}
                \Pro(Y>590)  &= \Pro(Z > 0.461)\\
                &= 0,5 - \Pro(0\leq Z< 0.461)\\
                &= 0,5 - 0.1772\\
                &= 0.3228
            \end{align*}
        }
		\item[b)] $\Pro(Y<616)$\\
		    \resposta{
	        \begin{align*}
                Z  &= \frac{616 - 580}{\sqrt{470}}\\
                &= 1.660556174
            \end{align*}
            \begin{align*}
                \Pro(Y<616)  &= \Pro(Z < 1.66)\\
                &= 0.5 + \Pro(0 \leq Z < 1.66)\\
                &= 0.5 + 0.4515\\
                &= 0.9515
            \end{align*}
        }
		\item[c)] $\Pro(550<Y<570)$\\
		    \resposta{
		    \begin{multicols}{2}
	        \begin{align*}
                Z_1  &= \frac{550 - 580}{\sqrt{470}}\\
                &= -1.383796812
            \end{align*}
            \begin{align*}
                Z_2  &= \frac{570 - 580}{\sqrt{470}}\\
                &= -0.461265604
            \end{align*}
            \end{multicols}
            \begin{align*}
                \Pro(550<Y<570) &= \Pro(-1.383<Z<-0.4612)\\
               % &= \Pro(-1.383<Z\leq0) + \Pro(0\leq Z<-0.4612)\\
                &= \Pro(0\leq Z \leq 1.383) - \Pro(0\leq Z<-0.4612)\\
                &= 0.4162-0.1772\\
                &= 0.239
            \end{align*}
        }
	\end{itemize}
	
	\item[\textbf{4.}] Considere 100 doadores escolhidos aleatoriamente de uma população onde a probabilidade de tipo A é 0,40? Qual a probabilidade de pelo menos 43 doadores terem sangue do tipo A?\\
	\resposta{n = 100\\
	$p = 0.4$\\
	Pelo TLC $n \geq 30$, aproximo para distribuição normal.\\
	$\E(X) = n\times p = 100\times0.4 = 40$\\
	$\V(X) = n\times p(1-p) = 100\times0.4\times0.6 = 24$\\
	$X \sim \N(\mu,\V(X)) \simeq X\sim\N(40; 24)$\\
	$Z = \frac{43-40}{\sqrt{24}} = \frac{3}{4.89} = 0.613$\\
	\begin{align*}
	    \Pro(A \geq 43) &= \Pro (Z \geq 0.613)\\
	    &= 0.5 - \Pro(0 \leq Z < 0.613)\\
	    &= 0.5 - 0.2291\\
	    &= 0.2709
	\end{align*}}
	
	\item[\textbf{5.}] A taxa de desemprego em certa cidade é de 10\%. É obtida uma amostra aleatória de 100 pessoas. Qual a probabilidade de uma amostra ter, pelo menos, 15 pessoas desempregadas.\\
	\resposta{
	n = 100\\
	p = 0.1\\
	$\E(X) = n\times p = 100\times0.1 = 10$\\
	$\V(X) = n\times p(1-p) = 100\times0.1\times0.9 = 9$\\
	$X \sim \N(\mu,\V(X)) \simeq X\sim\N(10; 9)$\\
	$Z = \frac{15-10}{\sqrt{9}} = \frac{5}{3} = 1.66$\\
	\begin{align*}
	    \Pro(A \geq 15) &= \Pro (Z \geq 1.66)\\
	    &= 0.5 - \Pro(0 \leq Z < 1.66)\\
	    &= 0.5 - 0.4515\\
	    &= 0.0485
	\end{align*}}
	
	\item[\textbf{6.}] Numa população, o peso dos indivíduos é uma variável aleatória X que segundo estudos anteriores segue o modelo normal com média 78 kg e desvio-padrão 10 kg. Uma pessoa é escolhida ao acaso nessa população. Determine a probabilidade de que seu peso:\\
	\resposta{$X \sim\N(78; 10^2)$}
		\begin{itemize}
		\item[a)] Seja maior que 60 kg;
	        \resposta{\begin{align*}
	            Z &= \frac{60 - 78}{10}\\
	            &= - 1.8
	        \end{align*}
	        \begin{align*}
	            \Pro(X > 60 ) &= \Pro(Z > -1.8)\\
	            &= 0.5 + \Pro(0 \leq Z \leq 1.8)\\
	            &= 0.5 + 0.4641\\
	            &= 0.9641
	        \end{align*}}
		\item[b)] Esteja entre 62kg e 72 kg
		    \resposta{
		    \begin{multicols}{2}
	        \begin{align*}
                Z_1  &= \frac{62 - 78}{10}\\
                &= -1.6
            \end{align*}
            \begin{align*}
                Z_2  &= \frac{72 - 78}{10}\\
                &= -0.6
            \end{align*}
            \end{multicols}
            \begin{align*}
                \Pro(62<X<72) &= \Pro(-1.6<Z<-0.6)\\
                &= \Pro(0\leq Z\leq 1.6) - \Pro(0 \leq Z \leq 0.6)\\
                &= 0.4452 - 0.2257\\
                &= 0.2195
            \end{align*}}
		\item[c)] Seja inferior a 90 kg
		    \resposta{\begin{align*}
	            Z &= \frac{90 - 78}{10}\\
	            &= 1.2
	        \end{align*}
	        \begin{align*}
	            \Pro(X < 90 ) &= \Pro(Z < 1.2)\\
	            &= 0.5 + \Pro(0 \leq Z < 1.2)\\
	            &= 0.5 + 0.3849\\
	            &= 0.8849
	        \end{align*}}
		\item[d)] Seja superior a 90 kg
		    \resposta{
		    \begin{align*}
	            \Pro(X > 90 ) &= \Pro(Z > 1.2)\\
	            &= 0.5 - \Pro(0 \leq Z < 1.2)\\
	            &= 0.5 - 0.3849\\
	            &= 0.1151
	        \end{align*}}
	\end{itemize}
	
\end{itemize}

\end{document}