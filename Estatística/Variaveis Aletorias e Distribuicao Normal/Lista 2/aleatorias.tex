\begin{center}
\textbf{\LARGE{Lista Variáveis aleatórias}}
\end{center}
\begin{itemize}
	\item[\textbf{1.}] Determine a probabilidade de obtermos exatamente 3 caras em 6 lances de uma moeda.\\
		\resposta{A = probabilidade de ser cara\\
		$\Pro(A) = \frac{1}{2}$ e $\Pro(A)^c = \frac{1}{2}$\\
		Usa-se a distribuição binomial:\\
		\begin{align*}
		    \Pro(3) &= \frac{6!}{3!(6-3)!}(\frac{1}{2})^{3}(\frac{1}{2})^{6-3}\\
		    & = \frac{720}{36}\times0.125\times 0.125\\
		    &= 20 \times 0.015625\\
		    &= 0.3125
		\end{align*}}
	\item[\textbf{2.}] Jogando-se um dado três vezes, determine a probabilidade de se obter um múltiplo de 3 duas vezes\\
		\resposta{$\Omega = \{3,6\}$\\
	Y = ser um múltiplo de 3\\
	Z = não é multilo de 3\\
	Y'= Ser múltiplo de 3 duas vez\\
	$\Pro(Y) = \frac{2}{6} = \frac{1}{3}$\\$\Pro(Z) =\frac{2}{3}$
	\begin{align*}
	    \Pro(Y') &= \frac{3!}{2!(3-2)!}(\frac{1}{3})^2\times(\frac{2}{3})^1\\
	     &= 3 \times \frac{1}{9}\times\frac{2}{3}\\
	     &= \frac{2}{9}
	\end{align*}}
	\ifx Se a ordem dos dados importar, temos 3 permutacoes: (YYZ), (YZY), (ZYY), ou seja,\\
	\begin{align*}
	    \Pro(Y') &= \frac{2}{27}\times\frac{2}{27}\times\frac{2}{27}\\
	    &= \frac{2}{9}
	\end{align*}
	\fi
	\item[\textbf{3.}] Dois times de futebol, A e B, jogam entre si 6 vezes. Encontre a probabilidade do time A.
	\resposta{A = time A ganhar um jogo\\
	Supondo que num jogo A possa ou ganhar ou perder ou empatar, temos que:
	$\Pro(A) = \frac{1}{3}$}
	\begin{itemize}
	    \item[\textbf{a.}] ganhar dois ou três jogos\\
	    \resposta{
	    B = ganhar dois jogos\\
	    C = ganhar três jogos \\
	    D = ganhar dois ou tres jogos\\
	   \begin{align*}
	    \Pro(B) &= (\frac{1}{3})^2\times(\frac{2}{3})^4\\
	    &= \frac{1}{9}\times\frac{16}{81}\\
	    &= \frac{16}{729}\\
	    &= 0.0219
	  \end{align*}
	  Se a ordem importar então:    $\Pro(B) = C^6_2 \times 0.0219 = 15\times 0.0219 = 0.329$
	  \begin{align*}
	    \Pro(C) &= (\frac{1}{3})^3\times(\frac{2}{3})^3\\
	    &= \frac{1}{27}\times\frac{8}{27}\\
	    &= \frac{8}{729}\\
	    &= 0.0109
	  \end{align*}
	  Se a ordem importar então:    $\Pro(C) = C^6_3 \times 0.0109 = 20\times 0.0109 = 0.2194$\\
	  Sendo assim:
	  \begin{align*}
	      \Pro(D) &= \Pro(B) + \Pro(C)\\
	               &= 0.0219 + 0.0109 & \text{ordem nao importa}\\
	               &= 0.0328\\
	               ou\\
	               &= 0.329 + 0.2194 & \text{ordem importa}\\ 
	               &=0.5484        
	  \end{align*}}
	    \item[\textbf{b.}] ganhar pelo menos um jogo. \\
	    \resposta{A probabilidade de ganhar pelo menos um jogo corresponde a todas as possibilidades possíveis com exceção dele perder todos os jogos:\\
	    E = ganhar pelo menos um jogo\\
	    \begin{align*}
	        \Pro(E) &= 1 - C^6_0 \times(\frac{1}{3}) ^0\times(\frac{2}{3})^6\\
	        &= 1 - \frac{64}{729}\\
	        &= \frac{665}{729}\\
	        &= 0.912
	    \end{align*}}
	\end{itemize}
\newpage
	\item[\textbf{4.}] A probabilidade de um atirador acertar o alvo é 2/3. Se ele atirar 5 vezes, qual a probabilidade de acertar exatamente 2 tiros?\\
	\resposta{
	A = acertar dois tipos
	\begin{align*}
	    \Pro(A) &= C^5_2 \times (\frac{2}{3})^2\times(\frac{1}{3})^3\\
	    &= 10 \times \frac{4}{9} \times \frac{1}{27}\\
	    &= \frac{40}{243}\\
	    &= 0.164
	\end{align*}} 
	\item[\textbf{5.}] Seis parafusos são escolhidos ao acaso da produção de certa máquina, que apresenta 10\% de peças defeituosas. Qual a probabilidade de serem defeituosos dois deles?\\
	%	\duvida{a máquina tem 10\% de pecas defeituosas, mas eu nao possuo a probabilidade de um parafuso ser defeituoso...}\\
	%\resposta{Supondo que um parafuso tenha 10\% de ser defeituoso, então:\\
	\resposta{Tendo 10\% como a probabilidade de um parafuso ser defeituoso, então:\\
	\begin{align*}
	    \Pro(2) &= C^6_2\times(\frac{1}{10})^2\times(\frac{9}{10})^4\\
	    &= \frac{15\times6561}{1000000}\\
	    &=0.0984
	\end{align*}}
	\item[\textbf{6.}] No fichário de um hospital, estão arquivados os prontuários dos de 20 pacientes, que deram entrada no PS apresentando algum problema cardíaco. Destes 5 sofreram infarto. Retirando-se uma amostra ao acaso de 3 destes prontuários, qual a probabilidade de que dois deles sejam de pacientes que sofreram infarto? Calcule o Valor esperado e a variância.\\
	\resposta{N = número de itens da população= 20\\
	M = numero de itens da populacao que sao considerados como ``sucesso" =  5\\
	n = numero de itens na amostra = 3\\
	k = 2 (para as combinacoes)\\
	\begin{align*}
	    \Pro(X = k) &= \frac{C_k^MC_{n-k}^{N-M}}{C^N_n}\\
	    \Pro(2) &= \frac{C_2^{5}C_{1}^{15}}{C^{20}_{3}}\\
	    &= \frac{10\times15}{1140}\\
	    &=0.131578
	\end{align*}
	O valor esperado é: $\E(x) = \frac{m\times n}{N} = \frac{5 \times 3}{20} = 0.75$\\
	\ifx
    \begin{tabular}{c|c}
        X & $\Pro(X)$ \\
        \hline\hline
        0 & $\frac{C_0^{5}C_{3}^{15}}{C^{20}_{3}}$\\
        1 & $\frac{C_1^{5}C_{2}^{15}}{C^{20}_{3}}$\\
        2 & $\frac{C_2^{5}C_{1}^{15}}{C^{20}_{3}}$\\
        3 & $\frac{C_3^{5}C_{0}^{15}}{C^{20}_{3}}$ \\
        \hline\hline\\
        $\E(x)$ &
    \end{tabular}
    \fi
	A variância é:
	\begin{align*}
	    \V(X) &= M\times p\times q \times\frac{N-n}{N-1}\\
	    &= 5\times\frac{5}{20}\times\frac{15}{20}\times\frac{17}{19}\\
	    &= \frac{6375}{7600}\\
	    &= 08388
	\end{align*}} 
	\item[\textbf{7.}] Suponha que selecionemos aleatoriamente 5 cartas baralho sem reposição de um de um maço ordinário de jogo de baralho. Qual é a probabilidade de obter exatamente 2 cartas de baralho vermelhas (isto é, copas ou ouros)?\\
	\resposta{Para essa questão estou considerando o baralho lusófono que possui 52 cartas sendo 13 de cada naipe. Sendo assim, as cartas vermelhas são 26 dentre as 52.\\
	N = número de itens da população= 52\\
	M = numero de itens da populacao que sao considerados como ``sucesso" =  26\\
	n = numero de itens na amostra = 5\\
	k = 2 (para as combinacoes)\\
	\begin{align*}
	    \Pro(X = k) &= \frac{C_k^MC_{n-k}^{N-M}}{C^N_n}\\
	    \Pro(2) &= \frac{C_2^{26}C_{3}^{26}}{C^{52}_{5}}\\
	    &= \frac{325\times 2600}{2598960}\\
	    &=0.32513
	\end{align*}} 
	\item[\textbf{8.}] Numa Loteria, um apostador escolhe 6 números de 1 a 54. Qual a probabilidade dele acertar 5 números?\\
	\resposta{
	N = número de itens da população= 54\\
	M = numero de itens da populacao que sao considerados como ``sucesso" =  6\\
	n = numero de itens na amostra = 6\\
	k = 5 (para as combinacoes)\\
	\begin{align*}
	    \Pro(X = k) &= \frac{C_k^MC_{n-k}^{N-M}}{C^N_n}\\
	    \Pro(2) &= \frac{C_5^{6}C_{1}^{48}}{C^{54}_{6}}\\
	    &= \frac{6\times48}{25827165}\\
	    &= \frac{288}{25827165}\\
	    &= 0.000011151
	\end{align*}}  
	\item[\textbf{9.}] Suponha-se que haja 50 pessoas, dos quais 34 são MULHERES e o restante são HOMENS. Extrai-se uma amostra aleatória de 15 pessoas, sem reposição. Qual a probabilidade de exatamente 5 pessoas serem do sexo FEMININO?
	\resposta{} 
	
	\item[\textbf{10.}] 
	    \item[\textbf{a.}] Exatamente duas estejam queimadas?
	    \item[\textbf{b.}] Pelo menos uma esteja boa? \item[\textbf{c.}] Pelo menos duas estejam queimadas?
	    \item[\textbf{d.}] O número esperado de lâmpadas queimadas? \item[\textbf{e.}] A variância do número de lâmpadas queimadas?
	
\end{itemize}